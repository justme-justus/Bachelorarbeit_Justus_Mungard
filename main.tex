% ========== Template by IC\M/T Institute of Creative\Media/Technologies =============
% ==========  M. Wagner and K. Blumenstein, N. Thuer 2017 ============= 
% ==========  Based on the LaTeX Thesis Template for the University of Applied Sciences St.Pölten by P. Lechner https://github.com/hrtlacek/ThesisTemplate-FH-StP        ============= 

%----------------------------------------------------------------
% TODO List

%----------------------------------------------------------------

%Dokumentklasse without end dot ;-)
\documentclass[a4paper,twoside,11pt, numbers=noenddot]{scrreprt}
\usepackage[left= 3.5cm,right = 3cm, bottom = 3.5 cm, top = 3 cm]{geometry}
\usepackage[onehalfspacing]{setspace}

% Standard Packages
\usepackage[utf8]{inputenc}

% ============= Settings for the Work =============

\def\workTitle{Isolierte und automatisierte Vernetzung von
Internetdiensten in einer hybriden
Cloud-Infrastruktur}
\def\subTitle{~}
\def\specialization{<Name of Masterclass>}
\def\studentFirstName{Justus Emil}
\def\studentLastName{Mungard}
\def\studentId{931 221}
\def\advisorPreTitle{Dipl.-Inform.}
\def\advisoFirstName{Corina}
\def\advisorLastName{Kopka}
\def\advisorCompanyPreTitle{Dr.}
\def\advisorCompanyFirstName{Roland}
\def\advisorCompanyLastName{Kaltefleiter}
\def\advisorPosTitle{}
\def\assessorPreTitle{}
\def\assessorFirstName{}
\def\assessorLastName{}
\def\assessorPosTitle{}
\def\place{Kiel}
\def\dateDay{26}
\def\dateMonth{05}
\def\dateYear{2021}

\newif\ifuseGermanVersion				    % <== DONT TOUCH THIS!!!
\newif\ifuseMasterInteractiveTechnologies 	% <== CAN'T TOUCH THIS!!! (da da dada)
\newif\ifuseMasterDigitalDesign	            % <== DONT TOUCH THIS!!!
\newif\ifuseMasterDigitalMediaProduction	% <== DONT TOUCH THIS!!!
\newif\ifuseMasterDigitalHealthCare			% <== DONT TOUCH THIS!!!
\newif\ifuseBachelorMediaTechnologiesOne    % <== DONT TOUCH THIS!!!
\newif\ifuseBachelorMediaTechnologiesTwo    % <== DONT TOUCH THIS!!!
\newif\ifuseBachelorSmartEngineeringOne     % <== DONT TOUCH THIS!!!
\newif\ifuseBachelorSmartEngineeringTwo     % <== DONT TOUCH THIS!!!

%***************************************************************************************
% To switch the version please use the comment "%" option :-). After a language change, you have to rebuild the whole project (in Overleaf --> recompile from scratch) 
\useGermanVersiontrue					    % German version
%\useGermanVersionfalse					    % English version
%***************************************************************************************
% To switch between the study programs use the comment option :-) 
% !!!ATTENTION: Only one has to be activated!!!
\useBachelorMediaTechnologiesOnetrue		% Bachelor #1 Media Technology
%\useBachelorMediaTechnologiesTwotrue		% Bachelor #2 Media Technology
%\useMasterInteractiveTechnologiestrue		% Master Interactive Technologies
%\useMasterDigitalDesigntrue		        % Master Digital Design
%\useMasterDigitalMediaProductiontrue		% Master Digital Media Production
%\useMasterDigitalHealthCaretrue			% Master Digital Health Care
% \useBachelorSmartEngineeringOnetrue		% Bachelor #1 Smart Engineering
%\useBachelorSmartEngineeringTwotrue		% Bachelor #2 Smart Engineering
%***************************************************************************************
%SVG
%\usepackage{svg}
%\usepackage{amsmath}

% ============= Packages =============

% Switch the Language
\ifuseGermanVersion
	\usepackage[ngerman]{babel}	% German
\else
	\usepackage[english]{babel} % English
\fi
\usepackage{csquotes}

\usepackage[T1]{fontenc}
%\usepackage{graphicx}
\usepackage{graphicx, subfigure}
%Set path for images
%\graphicspath{{img/}}
\usepackage{fancyhdr}
\usepackage{lmodern}
\usepackage{color}
\usepackage{transparent}
% Zitierstil
% Citation style
%\usepackage[comma,authoryear]{natbib}
%\usepackage{natbib}

%\usepackage[backend=biber, style=apa, citestyle=authoryear, sorting=nyt]{biblatex}

%\ifuseGermanVersion
%    \DeclareLanguageMapping{ngerman}{ngerman-apa}
%\else
%    \DeclareLanguageMapping{english}{english-apa}
%\fi

\usepackage[nottoc]{tocbibind}
%\addbibresource{biblatex.bib}

% zusätzliche Schriftzeichen der American Mathematical Society
\usepackage{amsfonts}
\usepackage{mathtools}

\usepackage[export]{adjustbox}

% BlockDiagram Drawing Package
% ---tikz
\usepackage{tikz}
\usetikzlibrary{positioning}
\usepackage{pgfplots}
\pgfplotsset{compat=1.10}
\usepackage{textcomp}

%====================================================
% Code Block Style Definition
%\documentclass{article}
\usepackage[utf8]{inputenc}

\usepackage{listings}
\usepackage{xcolor}

\definecolor{codegreen}{rgb}{0,0.6,0}
\definecolor{codegray}{rgb}{0.5,0.5,0.5}
\definecolor{codepurple}{rgb}{0.58,0,0.82}
\definecolor{backcolour}{rgb}{0.95,0.95,0.92}

\lstdefinestyle{mystyle}{
    backgroundcolor=\color{backcolour},   
    commentstyle=\color{codegreen},
    keywordstyle=\color{magenta},
    numberstyle=\tiny\color{codegray},
    stringstyle=\color{codepurple},
    basicstyle=\ttfamily\footnotesize,
    breakatwhitespace=false,         
    breaklines=true,                 
    captionpos=b,                    
    keepspaces=true,                 
    numbers=left,                    
    numbersep=5pt,                  
    showspaces=false,                
    showstringspaces=false,
    showtabs=false,                  
    tabsize=2
}

\lstset{style=mystyle}

\usepackage{lmodern} % for bold teletype font

%====================================================

%Package for using the [H] option on graphics to force them into place
\usepackage{float}

%iPython packages:
%\usepackage{graphicx} % Used to insert images
\usepackage{adjustbox} % Used to constrain images to a maximum size 
\usepackage{color} % Allow colors to be defined
%\usepackage{enumerate} % Needed for markdown enumerations to work
\usepackage{geometry} % Used to adjust the document margins
\usepackage{amsmath} % Equations
\usepackage{amssymb} % Equations
%\usepackage[mathletters]{ucs} % Extended unicode (utf-8) support
% \usepackage[utf8x]{inputenc} % Allow utf-8 characters in the tex document
\usepackage{fancyvrb} % verbatim replacement that allows latex
\usepackage{grffile} % extends the file name processing of package graphics 
                         % to support a larger range 
    % The hyperref package gives us a pdf with properly built
    % internal navigation ('pdf bookmarks' for the table of contents,
    % internal cross-reference links, web links for URLs, etc.)
\usepackage{hyperref}
\usepackage{longtable} % longtable support required by pandoc >1.10

%embedding of audio/video files etc.
% \usepackage{attachfile}
% \usepackage{movie15}
% \usepackage{media9}
% \usepackage{menukeys}

\usepackage[labelfont=it, labelsep=period, format=plain,justification=raggedright, singlelinecheck=false]{caption}
\captionsetup[figure]{justification=centering}
\definecolor{light-gray}{gray}{0.85}

% Switch between German and English based on the Settingx.tex. file
\usepackage{ifthen}

%\captionsetup[listing]{
%  labelsep = newline,
%  textfont = sc, 
%  name = LISTING, 
%  justification=justified,
%  singlelinecheck=false,%%%%%%% a single line is centered by default
%  labelsep=colon,%%%%%%
%  skip = \medskipamount}

% =============== BlockDiagram Drawing Config
\usetikzlibrary{shapes,arrows}

% Definition of blocks:
\tikzset{%
  block/.style    = {draw, thick, rectangle, minimum height = 3em,
    minimum width = 3em},
  sum/.style      = {draw, circle, node distance = 2cm}, % Adder
  input/.style    = {coordinate}, % Input
  output/.style   = {coordinate}, % Output
  mult/.style	  = {draw, isosceles triangle, minimum height=1cm, minimum width =1cm}
}
%mult/.style	  = {isosceles triangle, sharp corners, anchor=center, xshift=-4mm, minimum height=1.5cm, minimum width =0.05cm}
%isosceles triangle, fill=gray!25, minimum width=1.5cm

% Defining string as labels of certain blocks.
\newcommand{\suma}{\Large$+$}
\newcommand{\inte}{$\displaystyle \int$}
\newcommand{\derv}{\huge$\frac{d}{dt}$}
\newcommand{\conv}{\huge$\ast$}

% ============================================

% -- Settings für Code abbildungen
\usepackage{listings,lstautogobble}
\lstset{backgroundcolor=\color{light-gray},frame=single, framerule=0pt, showspaces=false, showtabs=false, numbers=left, numbersep=5pt, breaklines=false, autogobble=true, language=C++}

% Setze arial font
\usepackage[scaled]{helvet}
\renewcommand*{\familydefault}{\sfdefault}

% FH-grünBlau
\definecolor{FH}{rgb}{0.10, 0.57, 0.68}
% FH-grünBlau 2
\definecolor{FH2}{rgb}{0.0392, 0.666, 0.549}

% nicht einrücken nach Absatz
\setlength{\parindent}{0cm}

% Paragraph-Abstand
\setlength{\parskip}{0.3cm}

% ============= Kopf- und Fußzeile =============

%\renewcommand{\headrulewidth}{0.4pt}
%\renewcommand{\footrulewidth}{0pt}

\renewcommand{\chaptermark}[1]{\markboth{\thechapter~ #1}{}}

\fancypagestyle{icmt-fancy}{%
  \fancyhf{}% Clear header and footer
  \fancyhead[L]{\leftmark}
  \fancyfoot[R]{\thepage}% Custom footer
  \renewcommand{\headrulewidth}{0.4pt}% Line at the header visible
  \renewcommand{\footrulewidth}{0.0pt}% Line at the footer visible
}

% Redefine the plain page style
\fancypagestyle{plain}{%
  \fancyhf{}%
  \fancyfoot[R]{\thepage}%
  \renewcommand{\headrulewidth}{0pt}% Line at the header invisible
  \renewcommand{\footrulewidth}{0.0pt}% Line at the footer visible
}

% ============= Package Einstellungen & Sonstiges ============= 


%Besondere Trennungen
\hyphenation{De-zi-mal-tren-nung St-rei-fen-licht-scan-nern}

%römische Aufzählungen mit \RM{Zahl}
\newcommand{\RM}[1]{\MakeUppercase{\romannumeral #1}}

%Glossareintraege
\makeglossaries
\loadglsentries{Chapters/06_Glossar/entries}

% ============= Dokumentbeginn =============

\begin{document}

% Select the right main page ;-)
\ifuseGermanVersion
	
% setup page dimensions for titlepage
\newgeometry{left=2.4cm,right=2.4cm,bottom=2.5cm,top=2cm}

% force baselineskip and parindent
%\newlength{\tmpbaselineskip}
%\setlength{\tmpbaselineskip}{\baselineskip}
%\setlength{\baselineskip}{13.6pt}
%\newlength{\tmpparindent}
%\setlength{\tmpparindent}{\parindent}
%\setlength{\parindent}{17pt}

% first titlepage
\pagestyle{empty}

\begin{figure}[H]
\vspace*{-2.5cm}
\hspace*{2.5cm}
\includegraphics[keepaspectratio, width=1.31\textwidth, right]{TemplateElements/Fachhochschule_Kiel-logo.svg.png}
\end{figure}



\begin{center}

\vspace{1cm}

\begin{minipage}[t][5cm][s]{\textwidth}%
\centering
\Huge{{\color{FH2}{\fontsize{24}{30} \selectfont \workTitle\\}}}
\vspace{0.5cm}
\LARGE{{\color{FH2}{\fontsize{16}{24} \selectfont \subTitle\\}}}
\end{minipage}

\vspace{1cm}


\ifnum\ifuseBachelorMediaTechnologiesOne 1
\else\ifuseBachelorSmartEngineeringOne 1
\else0
\fi\fi
=1 
   	\LARGE{Bachelorarbeit}
\else
	\ifnum\ifuseBachelorMediaTechnologiesTwo 2
	\else\ifuseBachelorSmartEngineeringTwo 2
\else0
\fi\fi
=2
	\LARGE{Bachelorarbeit}
\else
	\ifuseMasterInteractiveTechnologies
		\LARGE{Masterarbeit}
	\else
	\ifuseMasterDigitalDesign
		\LARGE{Masterarbeit}
	\else
    \ifuseMasterDigitalMediaProduction
		\LARGE{Masterarbeit}
	\else
	\ifuseMasterDigitalHealthCare
		\LARGE{Masterarbeit}
    \else
        \LARGE{YOU HAVE TO CHOOSE THE PROGRAM TYPE IN THE SETTINGS!!!}
    \fi\fi\fi\fi
\fi\fi

% \ifuseBachelorMediaTechnologiesOne
% 	\LARGE{Research Paper}
% \else
% 	\ifuseBachelorSmartEngineeringOne
%     	\LARGE{Research Paper}
% \else
% 	\ifuseBachelorMediaTechnologiesTwo
% 		\LARGE{Bachelorarbeit}
% \else
% 	\ifuseMasterDigitalMediaTechnologies
% 		\LARGE{Masterarbeit}
% \else
% 	\ifuseMasterDigitalHealthCare
% 		\LARGE{Masterarbeit}
%     \else
%         \LARGE{YOU HAVE TO CHOOSE THE PROGRAM TYPE IN THE SETTINGS!!!}
%   	\fi
% \fi
% \fi
% \fi
% \fi




\vspace{1.3cm}
\ifuseBachelorMediaTechnologiesOne
	\fontsize{11pt}{15pt}\selectfont Bachelor Informationstechnologie\\
	Fachbereich Informatik und Elektrotechnik\\
Fachhochschule Kiel University of Applied Sciences\\  
\else
	\ifuseBachelorMediaTechnologiesTwo
		\fontsize{11pt}{15pt}\selectfont Bachelor-Studiengang \\
Fachhochschule Kiel\\  
\else
	\ifuseBachelorSmartEngineeringOne
    	\fontsize{11pt}{15pt}\selectfont Bachelor-Studiengang Smart Engineering\\
Fachhochschule Kiel\\ 
\else
	\ifuseMasterInteractiveTechnologies
		\fontsize{11pt}{15pt}\selectfont Ausgeführt zum Zweck der Erlangung des akademischen Grades\\
		\textbf{Dipl.-Ing. für technisch-wissenschaftliche Berufe}
\else
	\ifuseMasterDigitalDesign
		\fontsize{11pt}{15pt}\selectfont Ausgeführt zum Zweck der Erlangung des akademischen Grades\\
		\textbf{Dipl.-Ing. für technisch-wissenschaftliche Berufe}	
\else
    \ifuseMasterDigitalMediaProduction
		\fontsize{11pt}{15pt}\selectfont Ausgeführt zum Zweck der Erlangung des akademischen Grades\\
		\textbf{Dipl.-Ing. für technisch-wissenschaftliche Berufe}	
\else
	\ifuseMasterDigitalHealthCare
    	\fontsize{11pt}{15pt}\selectfont Ausgeführt zum Zweck der Erlangung des akademischen Grades\\
		\textbf{Master of Science in Engineering (MSc)}
    \else
        \LARGE{YOU HAVE TO CHOOSE THE PROGRAM TYPE IN THE SETTINGS!!!}
\fi\fi\fi\fi\fi\fi\fi

\vspace{4mm}

\ifuseMasterInteractiveTechnologies
	am Masterstudiengang Interactive Technologies an der\\ 
Fachhochschule St. Pölten, Masterklasse \specialization
\else
    \ifuseMasterDigitalDesign
	am Masterstudiengang Digital Design an der\\ 
Fachhochschule St. Pölten, Masterklasse \specialization
\else
    \ifuseMasterDigitalMediaProduction
	am Masterstudiengang Digital Media Production an der\\ 
Fachhochschule St. Pölten, Masterklasse \specialization
\else
	\ifuseMasterDigitalHealthCare
		am Masterstudiengang Digital Healthcare\\ 
an der Fachhochschule St. Pölten
    \else
        
  	\fi\fi\fi\fi

\vspace{1cm}
\ifuseBachelorMediaTechnologiesOne
	Vorgelegt von:
    
\else
	Ausgeführt von:\\ 
\fi
\fontsize{15pt}{15pt}\selectfont
\textbf{\studentFirstName\ \studentLastName} \\
\fontsize{11pt}{15pt}\selectfont
\studentId

\vspace{1cm}
\ifuseBachelorMediaTechnologiesOne
	\begin{tabular}{lll}
    Erstprüferin: & & \advisorPreTitle\ \advisoFirstName\ 		\advisorLastName\\
    Zweitprüfer: & & \advisorCompanyPreTitle\ \advisorCompanyFirstName\ \advisorCompanyLastName\\
    \end{tabular}
\else
	\ifuseBachelorMediaTechnologiesTwo
		\begin{tabular}{lll}
        Betreuer/in: & & \advisorPreTitle\ \advisoFirstName\ \advisorLastName, \advisorPosTitle\\
        %Zweitbegutachter/in: & & [Titel Vorname Zuname]
		\end{tabular}
\else
\begin{tabular}{lll}
Betreuer/in: & \advisorPreTitle\ \advisoFirstName\ \advisorLastName, \advisorPosTitle\\
Zweitbetreuer/in: & \assessorPreTitle\ \assessorFirstName\ \assessorLastName, \assessorPosTitle\\
\end{tabular}

\fi
\fi

\vspace{1cm}


\large{\place, \dateDay.\dateMonth.\dateYear}


\end{center}

\restoregeometry
\else
	
% setup page dimensions for titlepage
\newgeometry{left=2.4cm,right=2.4cm,bottom=2.5cm,top=2cm}

% force baselineskip and parindent
%\newlength{\tmpbaselineskip}
%\setlength{\tmpbaselineskip}{\baselineskip}
%\setlength{\baselineskip}{13.6pt}
%\newlength{\tmpparindent}
%\setlength{\tmpparindent}{\parindent}
%\setlength{\parindent}{17pt}

% first titlepage
\pagestyle{empty}

\begin{figure}[H]
\vspace*{-2.5cm}
\hspace*{2.5cm}
\includegraphics[keepaspectratio, width=1.4\textwidth, right]{TemplateElements/fhLogo3.png}
\end{figure}



\begin{center}

\vspace{1cm}

\begin{minipage}[t][5cm][s]{\textwidth}%
\centering
\Huge{{\color{FH2}{\fontsize{24}{30} \selectfont \workTitle\\}}}
\vspace{0.5cm}
\LARGE{{\color{FH2}{\fontsize{16}{24} \selectfont \subTitle\\}}}
\end{minipage}

\vspace{1cm}

\ifuseBachelorMediaTechnologiesOne
	\LARGE{Research Paper}
\else
	\ifuseBachelorMediaTechnologiesTwo
		\LARGE{Bachelor Thesis}
\else
	\ifuseMasterInteractiveTechnologies
		\LARGE{Master Thesis}
\else
	\ifuseMasterDigitalDesign
		\LARGE{Master Thesis}
\else
    \ifuseMasterDigitalMediaProduction
		\LARGE{Master Thesis}
\else
	\ifuseMasterDigitalHealthCare
		\LARGE{Master Thesis}
    \else
        \LARGE{YOU HAVE TO CHOOSE THE PROGRAM TYPE IN THE SETTINGS!!!}
  	\fi
\fi
\fi
\fi\fi\fi
  
\vspace{1.3cm}
\ifuseBachelorMediaTechnologiesOne
	\fontsize{11pt}{15pt}\selectfont Bachelor Course on Media Technology\\
at St. Pölten University of Applied Sciences\\  
\else
	\ifuseBachelorMediaTechnologiesTwo
		\fontsize{11pt}{15pt}\selectfont Bachelor Course on Media Technology\\
at St. Pölten University of Applied Sciences\\  
\else
	\ifuseMasterInteractiveTechnologies
		\fontsize{11pt}{15pt}\selectfont For attainment of the academic degree of\\
		\textbf{Dipl.-Ing. für technisch-wissenschaftliche Berufe}
\else
    \ifuseMasterDigitalDesign
		\fontsize{11pt}{15pt}\selectfont For attainment of the academic degree of\\
		\textbf{Dipl.-Ing. für technisch-wissenschaftliche Berufe}
\else
    \ifuseMasterDigitalMediaProduction
		\fontsize{11pt}{15pt}\selectfont For attainment of the academic degree of\\
		\textbf{Dipl.-Ing. für technisch-wissenschaftliche Berufe}
\else
	\ifuseMasterDigitalHealthCare
    	\fontsize{11pt}{15pt}\selectfont For attainment of the academic degree of\\
		\textbf{Master of Science in Engineering (MSc)}
    \else
        \LARGE{YOU HAVE TO CHOOSE THE PROGRAM TYPE IN THE SETTINGS!!!}
  	\fi
\fi
\fi
\fi\fi\fi

\vspace{4mm}
 
\ifuseMasterInteractiveTechnologies
	in the Masters Course Interactive Technologies at St. Pölten\\ 
University of Applied Sciences, Masterclass \specialization
\else
    \ifuseMasterDigitalDesign
	in the Masters Course Digital Design at St. Pölten\\ 
University of Applied Sciences, Masterclass \specialization
\else
    \ifuseMasterDigitalMediaProduction
	in the Masters Course Digital Media Production at St. Pölten\\ 
University of Applied Sciences, Masterclass \specialization
\else
	\ifuseMasterDigitalHealthCare
		in the Masters Course Digital Healthcare\\ 
at St. Pölten University of Applied Sciences
    \else
  	\fi
\fi\fi\fi





\vspace{1cm}

Submitted by:\\ 
\fontsize{15pt}{15pt}\selectfont
\textbf{\studentFirstName\ \studentLastName} \\
\fontsize{11pt}{15pt}\selectfont
\studentId

\vspace{1cm}
\ifuseBachelorMediaTechnologiesOne
	\begin{tabular}{lll}
    Advisor: & & \advisorPreTitle\ \advisoFirstName\ 		\advisorLastName, \advisorPosTitle\\
    %Zweitbegutachter/in: & & [Titel Vorname Zuname]
    \end{tabular}
\else
	\ifuseBachelorMediaTechnologiesTwo
		\begin{tabular}{lll}
        Advisor: & & \advisorPreTitle\ \advisoFirstName\ \advisorLastName, \advisorPosTitle\\
        %Zweitbegutachter/in: & & [Titel Vorname Zuname]
		\end{tabular}
\else
  \begin{tabular}{lll}
  Advisor: & \advisorPreTitle\ \advisoFirstName\ \advisorLastName, \advisorPosTitle\\
  Second Advisor: & \assessorPreTitle\ \assessorFirstName\ \assessorLastName, \assessorPosTitle\\
  \end{tabular}

\fi
\fi

\vspace{1cm}


\large{\place, \dateDay.\dateMonth.\dateYear}


\end{center}

\restoregeometry
\fi

% \part im Inhaltsverzeichnis nicht nummerieren
\makeatletter
\let\partbackup\l@part
\renewcommand*\l@part[2]{\partbackup{#1}{}}

%Seitennummerierung neu beginnen, Zahlen [arabic], röm.Zahlen [roman,Roman], Buchstaben [alph,Alph]
\pagenumbering{Roman}

\newpage

\chapter*{Eidesstattliche Erklärung}
\label{ch:erklaerung}

\begin{flushleft}
Ich versichere mit meiner eigenhändigen Unterschrift, dass ich die vorliegende Abschlussarbeit selbständig und ohne unzulässige, fremde Hilfe angefertigt habe und dass ich alle von anderen Autor*innen wörtlich übernommenen Stellen wie auch die an die Gedankengänge und Strukturen anderer Autor*innen eng angelehnten Ausführungen meiner Arbeit besonders gekennzeichnet und die entsprechenden Quellen angegeben habe.
\end{flushleft}

\begin{flushleft}
Weiterhin versichere ich, dass die vorliegende Abschlussarbeit noch keiner Prüfungsbehörde vorgelegen hat.
\end{flushleft}

Ort: \hrulefill\enspace Datum:	\hrulefill\enspace Unterschrift: \hrulefill
\\[3.5cm]
\newpage

\chapter{Abstract}
Unternehmen verlagern zunehmend digitale Prozesse, die ehemals auf einer eigenen, physischen Infrastruktur abgebildet wurden, hin zu \textit{Public Cloud}-Providern wie Amazon Web Services und Microsoft Azure. Die Cloud im Allgemeine verspricht ein schnelles Deployment und hohe Verfügbarkeiten von Internetdiensten. Weiterhin können diese Dienste dynamisch zum Anfrageaufkommen skaliert werden. Public Clouds bieten Unternehmen den Vorteil, dass keine physische Infrastruktur vonnöten ist, um diese Dienste anzubieten. Ein kompletter Umzug findet dabei aber oftmals nicht statt und es verbleibt Infrastruktur in einer \textit{Private Cloud}, z.B. um die Speicherung von sensitiven Daten weiterhin kontrollieren zu können.\\
Das Ziel der vorliegenden Arbeit ist zu zeigen, wie eine Vernetzung von Public und Private Cloud ermöglicht werden kann, um einen vom Internet isolierten Datenaustausch via IPv4 zu gewährleisten. Hauptsächlicher Anwendungsbereich sind kleine bis mittelgroße Skalierungen.\\
Um diese Forschungsfrage zu beantworten, wurden zwei Use-Cases definiert, welche praxisrelevante Hybrid Cloud-Szenarien widerspiegeln. Use-Case 1 stellt die grundlegende IPv4-Kommunikation (\glqq Ende-zu-Ende\grqq{}) zwischen verschiedenen Cloud-Plattformen dar, während in Use-Case 2 (\glqq Roadwarrior-VPN\grqq{}) eine latenzarme Verbindung von Endgeräten zu der nächstgelegenen Cloud-Plattform erfolgen soll. Durch geeignete Werkzeuge soll dies automatisiert geschehen.\\
Mit Hilfe geeigneter Evaluationskriterien konnte erfolgreich gezeigt werden, dass eine technische Machbarkeit der aufgezeigten Szenarien grundsätzlich gegeben ist. Technische Hürden existieren und werden ausführlich beleuchtet.\\
Weiterführende Forschungsthemen im Hybrid Cloud-Umfeld wie Bandbreitenoptimierung sind erdenklich und werden im Ausblick der Arbeit näher erläutert.\\
%Inhaltsverzeichnis
\pagestyle{plain}
\tableofcontents

\newpage
%Seitennummerierung neu beginnen, Zahlen [arabic], röm.Zahlen [roman,Roman], Buchstaben [alph,Alph]
\pagenumbering{arabic}

% pagestyle für gesamtes Dokument aktivieren
\pagestyle{icmt-fancy}
\newpage


%SASE? https://en.wikipedia.org/wiki/Secure_Access_Service_Edge
%keine KMU => kleine und mittlere Skalierungen
\chapter{Einleitung}

Viele Kunden der NetUSE AG wechseln auf Cloud-basierte Dienste, um ehemals hausinterne IT-Infrastruktur dorthin zu verlagern bzw. zu erweitern. Auf der technischen Ebene spricht dafür oftmals, dass passende Infrastruktur für einen Dienst \textit{on premise}\footnote{\url{https://web.archive.org/web/20200515145106/https://www.cloudcomputing-insider.de/was-ist-on-premises-a-623402/}} nicht zur Verfügung steht. \textit{Entscheider} sehen in der Cloud eine Möglichkeit der Kosteneinsparung zusammen mit einer besseren Aufschlüsselung, welche Kosten für den Betrieb eines Dienstes verursacht werden. Kosten bspw. für den Aufbau und Betrieb können sich in der konventionellen Infrastruktur über verschiedene Teams erstrecken. Das wären bspw. bei der NetUSE AG: Server-Einbau und -Installation (Team \glqq{Infrastruktur}\grqq{}), Netzwerkanbindung (Team \glqq{Netzwerk}\grqq{}), Pflege von Firewall-Regeln (Team \glqq{Security}\grqq{}).\\
In der Cloud werden diese Schritte zusammengefasst und per Knopfdruck können Maschinen deployed\footnote{\url{https://web.archive.org/web/20190430104856/https://www.cloudcomputing-insider.de/was-ist-deployment-a-614153/}} werden. Dies passiert in größeren Skalierungen automatisiert, z.B. Skript-gesteuert oder mit Hilfe von Automatisierungs-Frameworks wie Ansible\footnote{\url{https://web.archive.org/web/20200917042646/https://docs.ansible.com/ansible/latest/index.html}} oder Terraform\footnote{\url{https://web.archive.org/web/20201003060715if\_/https://www.terraform.io/intro/index.html}}, wobei Cloud-Anbieter die passenden Schnittstellen zur Orchestrierung\footnote{\url{https://web.archive.org/web/20201027094107/https://www.redhat.com/de/topics/automation/what-is-orchestration}} anbieten.\\
Cloud-Dienste bringen damit einen Paradigmenwechsel in die IT-Landschaft: Server, auf denen gewünschte Dienste installiert sind, sind nicht mehr vor Ort beim Kunden oder bei einem Managed Service Provider\footnote{\url{https://web.archive.org/web/20201010111438/https://www.it-business.de/was-ist-ein-managed-service-provider-msp-a-577911/}} angesiedelt. Typische Prozesse wie die Grundinstallation eines Betriebssystems erübrigen sich. Analog dazu sind viele Software-Lösungen bereits über einen Cloud-Marketplace erhältlich, so dass diese gleich mit der Initialisierung der Cloud-Instanz starten.\\
Ebenso gibt es in der Cloud keine klassische Netzwerk-Infrastruktur: Broadcast-fähige Layer-2-Domänen, welche ursprünglich notwendig waren, um lokale Netzwerkteilnehmer (ARP [IPv4] bzw. Neighbor Discovery [IPv6]) zu finden, werden nicht mehr benötigt: Die Cloud \textit{kennt} naturgemäß alle Teilnehmer und es ist nach Belieben möglich, Konnektivität zwischen diesen zu erlauben bzw. zu verbieten.\\
Netzwerke in der Cloud sind Software-Defined (SDN)\footnote{\url{https://web.archive.org/web/20151002095242/http://www.heise.de/ix/heft/Losgelassen-2556771.html}}, was zur Folge hat, dass gewohnte Werkzeuge eines Netzwerk-Ingenieurs wegfallen, während gleichzeitig neue dazu gekommen sind, um Netzwerke (wieder-)herzustellen.\\
Die NetUSE AG sieht zur Zeit zwei \textit{public Cloud}-Anbieter für ihr Geschäft im Fokus: Amazon Web Services (AWS) und Microsoft Azure. Weiterhin unterstützt die Firma Kunden beim Aufbau einer \glqq{private Cloud}\grqq{} mit Red Hat OpenShift. Unterhält ein Unternehmen mehr als eine Cloud-Plattform, wird dies als Hybrid-Cloud bezeichnet.\footnote{\url{https://web.archive.org/web/20201025235403/https://www.redhat.com/de/topics/cloud-computing/what-is-hybrid-cloud}}\\
Eine Vernetzung zwischen den Plattformen ist nicht trivial, insofern man nicht mit öffentlicher IP-Adressierung jeder Maschine in der Cloud arbeiten möchte.
Die Bachelor-Arbeit zielt darauf ab, wie ein möglichst generisches \textit{Netzwerk-Overlay}\footnote{\url{https://web.archive.org/web/20180610164252/https://link.springer.com/referenceworkentry/10.1007\%2F978-0-387-39940-9\_1231}} zwischen verschiedenen Cloud-Anbietern herzustellen ist. Dies soll automatisiert passieren, sodass eine IP-Konnektivität neuer Maschinen und \glqq{Cloud-Domänen}\grqq{} über die Anbietergrenzen hinweg in kurzer Zeit gegeben ist. Die Konfigurationen sollen möglichst über die bereits verfügbaren Schnittstellen der Anbieter stattfinden. Sobald die Netzwerkumgebung zur Verfügung steht, soll anhand typischer Use-Cases von kleinen und mittelständischen Unternehmen gezeigt werden, wie typische (Server-)Dienste und Fail-Over-Szenarien\footnote{\url{https://web.archive.org/web/20201202083146/https://www.itwissen.info/failover-Failover.html}} in einem Cloud-Netzwerk abgebildet werden können.

\chapter{Inhalt und Ziel}
\section{Ziel}

Es existieren bereits proprietäre Lösungen am Markt, um eine Vernetzung über Cloud-Anbieter hinweg zu ermöglichen. Sie fallen in die Kategorie \textit{Software-Defined Wide Area Network (SD-WAN)} und sind nicht interoperabel zu SD-WAN-Lösungen anderer Hersteller.\\
Ziel der Arbeit wird sein, ein möglichst agnostisches Netzwerk-Overlay zwischen verschiedenen Cloud-Plattformen herzustellen. Es soll gezeigt werden, wie neue Cloud-Instanzen automatisiert in das Netzwerk des Inhabers eingebunden werden, so dass sie für alle berechtigten Teilnehmer über eine vom Internet isolierte Netzwerk-Infrastruktur erreichbar sind. Dies soll möglichst mit freier, offener Software und RFC-konformen Protokollen\footnote{\url{https://web.archive.org/web/20200912183838/https://www.ietf.org/standards/rfcs/}} geschehen.\\
Interessant ist diese Lösung bspw. für Unternehmen, welche die Lizenzkosten für proprietäre Lösungen sparen wollen oder bevorzugt auf offene Systeme bzw. Standards setzen, welche eine hohe Interoperabilität zu anderen Systemen versprechen. Auch können so auf eine schnelle Art und Weise neue Cloud-Netzwerke in die konventionelle Firmen-Infrastruktur eingebunden werden.

\section{Inhalt}
%Es soll ein Katalog erstellt werden, welcher typische Anforderungen von kleinen und mittelständischen Kunden der NetUSE AG auflistet. Dieser umfasst vsl. Punkte der Skalierbarkeit, Orchestration, Performanz und Sicherheit hinsichtlich einer Cloud-Vernetzung.\\
%Da die NetUSE AG jahrelanger Partner des Netzwerk-Ausrüsters Cisco ist, soll untersucht werden, welche Lösungen in diesem Feld bereits von Cisco angeboten werden. Dabei wird der entworfene Anforderungskatalog zu Hilfe genommen.\\
%Aus diesem Erkenntnisgewinn soll ein Design für eine offene Lösung erarbeitet und in einem Proof-of-Concept manifestiert werden. Um eine Automatisierung zu erlangen, werden evtl. Programmierarbeiten erforderlich.\\
Es sollen verschiedene Use-Cases erarbeitet werden, die im Kontext einer Cloud-Vernetzung geeignet erscheinen, um verschiedene Netzwerkdienste (redundant) anzubieten. Grundlegend wird davon ausgegangen, dass ein Kunde diese Dienste sowohl bei sich vor Ort als auch bei den Cloud-Providern Amazon Web Services und Microsoft Azure in Anspruch nehmen möchte, um eine Hochverfügbarkeit bzw. einen Lastenausgleich zu erreichen.\\
Im ersten Schritt muss dafür ein geeignetes Design für die Vernetzung zwischen den verschiedenen Standorten gefunden werden. Dieses Design soll mit Hilfe geeigneter technischer Werkzeuge umgesetzt werden und dient im Anschluss als Grundlage für die Abbildung der vorher definierten Use-Cases.
Diese technisch abgebildeten Use-Cases sind im Einzelnen als Proof-of-Concept anzusehen und sollen abschließend in Bezug auf im Voraus formulierte Anforderungen und Kriterien analysiert und bewertet werden.
Es ist davon auszugehen, dass im Rahmen der Umsetzung der Autor mit diversen technischen Problemen konfrontiert sein wird. Die Analyse und Diskussion dieser Problemstellungen soll zu geeigneten Lösungsansätzen führen.

\subsection{Aspekte, die die Bachelor-Arbeit herausstellen soll}
\begin{itemize}
    \item Kein Vendor Lock-In\footnote{\url{https://web.archive.org/web/20200919040133/https://journalofcloudcomputing.springeropen.com/articles/10.1186/s13677-016-0054-z}} des Cloud-Anbieters, da das Netzwerk über mehrere Cloud-Anbieter gespannt wird
    \item Möglichkeit der Hochverfügbarkeit und des Lastenausgleichs für Netzwerkdienste über mehrere Cloud-Plattformen hinweg
    \item Projekte oder Dienste, die eine bestimmte Cloud-Plattform erfordern, können einfach für weitere (Cloud-)Standorte verfügbar gemacht werden
    \item Latenz- oder Bandbreitenanforderungen können durch eine Hybrid-Cloud-Strategie u.U. besser eingehalten werden
    \item Neue Cloud-Netzwerke können dynamisch und automatisiert aufgebaut werden und sind im Anschluss für alle berechtigten Teilnehmer verfügbar
\end{itemize}

\subsection{Fragestellungen, auf die diese Arbeit eingehen wird}
\begin{itemize}
    \item Mit Hilfe welcher Technologien und Systemkomponenten lässt sich so eine Lösung aufbauen?
    \item Welche Schnittstellen werden bei den jeweiligen Cloud-Plattformen angeboten, um eine automatisierte Vernetzung zu bewerkstelligen?
    \item Welche Werkzeuge können mit diesen Schnittstellen genutzt werden?
    \item Welche Netzwerk-Topologien bieten sich an, um mehrere Cloud-Plattformen miteinander zu verbinden?
    \item Welche Netzwerk-Protokolle bieten sich an, um mehrere Cloud-Plattformen miteinander zu verbinden?
    \item Wie kann eine hohe Sicherheit im Netzwerkverkehr zwischen den Plattformen gewährleistet werden, um fremde Zugriffe zu verhindern?
    \item Sind Bandbreiten- und Latenzanforderungen in typischen Szenarien erfüllbar? Wie kann u.U. ein Lastenausgleich herbeigeführt werden?
\end{itemize}

\subsection{Abgrenzung der Arbeit}

\begin{itemize}
%Nicht Overlay... Abstraktion, um per IPv4 eine Ende-zu-Ende-Konnektivität herzustellen
\item Es wird herausgearbeitet, wie ein IPv4-Overlay geschaffen werden kann. Dies soll mit privaten RFC1918-Adressen geschehen und dies automatisiert evtl. unter Zuhilfenahme eines IP Adressmanagements (IPAM)\footnote{\url{https://www.ip-insider.de/was-ist-ipam-ip-address-management-a-771675/}}. Öffentliche IPv4-Netzwerke werden nicht berücksichtigt.
\item IPv6-Adressierung wird in dieser Arbeit nicht berücksichtigt: IPv6-Unterstützung ist auf den Cloud-Plattformen teilweise eingeschränkt und Kunden nutzen meist noch IPv4 für interne Adressierungen.
\item Broadcast- und Multicast-Anwendungen werden in den Use-Cases nicht berücksichtigt. Auch hier gibt es nur eingeschränkte Unterstützung seitens der Cloud-Dienstleister.
%\item Hochverfügbarkeit von Internetdiensten (bspw. ein MySQL-Cluster) über mehrere Cloud-Plattformen hinweg ist nicht Teil der Bachelor-Arbeit.
\item Weitere Cloud-Anbieter wie Google oder IBM werden nicht berücksichtigt. Das Hauptaugenmerk liegt auf Amazon Web Services, Microsoft Azure und der Integration von \textit{klassischer} Netzwerk-Infrastruktur, wie man sie bei typischen kleinen und mittelständischen Unternehmen der NetUSE AG vorfindet.
\item Typische On-Premise Cloud-Lösungen wie OpenStack oder OpenShift werden auf Grund ihrer hohen Komplexität in der Bachelor-Arbeit nicht berücksichtigt. Solche Standorte sollen in der Bachelor-Arbeit durch eine Abstraktion dargestellt werden. Darauf wird in der Bachelor-Arbeit näher eingangen.
\item Vernetzungskonzepte \textit{innerhalb} eines Cloud-Netzwerkes sind nicht Teil der Bachelor-Arbeit. Sie werden aber evtl. zur Bearbeitung bestimmter Use-Cases benötigt. Ggf. werden diese Konzepte dann genauer erläutert.
\item Die zu bearbeitenden Use-Cases und Fail-Over-Szenarien werden im Verlauf der Bachelor-Arbeit aufgeschlüsselt und sind nicht Teil dieses Proposals.
% auch SLA genannt
\item Ausblick Härtung Encryption..
\item Auf Bandbreitenmessungen wird in dieser Arbeit bewusst verzichtet, da Bandbreiten von zu vielen Faktoren abhängig ist, es werden allerdings im Ausblick ggf. Optimierungen dargelegt...
\item Entgegen des initialen Proposals zielt die Arbeit nicht mehr auf kleine und mittelständische Unternehmen, sondern auf kleine bis mittelgroße \textbf{Skalierungen} ab. Es hat sich während der praktischen Bearbeitung der Use-Cases herausgestellt, dass die genutzten Technologien so robust sind, dass ein Einsatz auch im \glqq größeren \textbf{technischen} Rahmen\grqq{} machbar ist. Aus technischer Sicht stellt sich nicht die Frage, wie \glqq groß\grqq{} ein Kunde ist, sondern welche Anforderungen (Latenz, Bandbreite, Uptime, etc.) in der Praxis tatsächlich vorherrschen. Es wird daher zum Ende der Arbeit noch ein Ausblick gegeben, wie beliebig große Skalierungen mit schärferen Performance-Kriterien, neben den erarbeiteten Lösungsvorschlägen, auch bedient werden können.
\item Auf Grund der zuvor genannten Punkte ist "letzer Punkt Fragestellung" nicht im vollen Maße mit dieser Arbeit zu erfüllen und könnte eigene Arbeit werden (s. Ausblick)
\item Ende-zu-Ende-Konnektivität gegeben (wegen Amazon NAT z.B.)
%Kein Best Practice Security, DH etc...

%BFD, SLA, QoS, VNET / VPC Peering / Microsoft SDWAN / TG?

\end{itemize}

\iffalse
Serifen Schrit für Programmnamen?
Teilweise gekürzt mit [...] oder Funktionsnamen der Einfachheit umbenannt oder vereinfacht dargestellt
RFC-Doku-Addressen
Getestet mit Terraform Version...
Abkürzungen AWS, VPC werden einmal erläutert, danach Schicht im Schacht
Rein technischer Natur, es wird nicht auf Kostenoptimierung eingangen -> dafür ist Cloud-Costs zu komplex
Abgrenzung: alle Public-IP und Private-IP beziehen sich auf IPv3!!!
Kommandos sind mit Dollar markiert. Sie werden genutzt, wenn ein Filterausdruck erkennbar sein soll, oftmals mit grep
Typische Schalter in dieser Arbeit:
-A n: Zeige auch die nächsten n Zeilen nach einem Treffer
-B n: Zeige auch die vorherigen n Zeilen vor einem Treffer
-o Zeige ausschließlich den Treffer, nicht die komplette Zeile, oftmals im Zusammenspiel mit -E
-E Suche mit Hilfe von regulären Ausdrücken (Regex)
-v ignoriere Zeilen, die das Pattern beinhalten (invertiere)
Linux-Grundlagen werden vorausgesetzt, iptables, cat, head, tail, 
Alle Zeichnungen wurden draw.io gemacht. Auch die Shapes entstammen draw.io
Monospace für Programmaufrufe
Unterstrichen für Dateinamen
Alle Codebeispiele finden sich im Github
\fi
\chapter{Technische Grundlagen} \label{Technische_Grundlagen}

\section{Internet Protocol Version 4 (IPv4)} \label{ipv4}
\subsection{Allgemeines und Adressierungsschema}
Das Internet Protocol ist im OSI-Modell auf Schicht 3 (Network-Layer) anzusiedeln. Es dient der \textit{Paketierung} von Daten und anschließenden Ende-zu-Ende-Vermittlung dieser \textit{Pakete} zwischen Netzwerk-Systemen. Dafür notwendig ist eine Adressierung: eine IPv4-Adresse besteht aus einer 32 Bit-Zahl, welche typischerweise dezimal dargestellt wird in so genannter \glqq dotted decimal\grqq{}-Schreibweise. Dafür wird die Binärzahl in vier Oktette mit einem Punkt aufgeteilt und die jeweiligen Oktette in dezimal umgerechnet.

00000001000000100000001100000100 (32-Bit Integer) ->  00000001.00000010.00000011.00000100 (dotted binary) -> 1.2.3.4 (dotted decimal)

Mit Hilfe von Netzmasken wird die Netzwerkgröße bestimmt. Der Host mit der IP-Adresse 1.2.3.4 und einer Netzmaske von bspw. 255.255.255.0. Diese Netzmaske entspricht in dotted binary:

11111111.11111111.11111111.00000000

Die ersten 24 Bit sind für das Netzwerk reserviert (Network bits), die letzte 8 Bit sind Host bits. Dieses Netzwerk kann somit theoretisch aus $2^8 = 256$ Hosts bestehen, wobei die erste und letzte Adresse reserviert sind (Netzwerkadresse und Broadcast-Adresse).

Heutzutage üblich ist die so genannte CIDR-Notation, welche IP-Addresse und Subnetzmaske vereint. Wenn man die erwähnten Beispiele aus Adresse und Subnetzmaske zusammenführt, erhält man die IP-Adresse in CIDR-Notation: 1.2.3.4/24. /24 entspricht dabei den gesetzten Network bits.

Mit Hilfe der IP-Adresse und Netzmaske lässt sich ermitteln, ob sich ein Host im gleichen \textit{Subnetz} befindet. Dafür berechnet man die Netzwerkadresse per logisch-AND:

\begin{lstlisting}[label=local-ip-address-AND-subnet,caption=Blub]
    00000001.00000010.00000011.00000100
AND 11111111.11111111.11111111.00000000
    00000001.00000010.00000011.00000000 => Netzwerkadresse: 1.2.3.0/24
\end{lstlisting}
    
Hat man nun zwei Hosts 1.2.3.240 und 8.9.10.23, mit denen man eine IP-Verbindung aufbauen will, so prüft der Initiator zuerst, ob das jeweilige Netz \textit{link-lokal} erreichbar ist, also im gleichen IP-Netzwerk wie der Initiator liegt.

Für 1.2.3.240:

\begin{lstlisting}[label=local-ip-address-AND-subnet-same,caption=Blun]
... 00000001.00000010.00000011.11110000
AND 11111111.11111111.11111111.00000000
  = 00000001.00000010.00000011.00000000 => Netzwerkadresse: 1.2.3.0/24 (gleiche Netzwerkadresse wie oben, link-lokal erreichbar)
\end{lstlisting}

Für 8.9.10.23:

\begin{lstlisting}[label=local-ip-address-AND-subnet-different,caption=Blub]
    00001000.00001001.00001010.00010111
AND 11111111.11111111.11111111.00000000
    00001000.00001001.00001010.00010111 => Netzwerkadresse: 8.9.10.0/24 (andere Netzwerkadresse: das Paket muss \textit{geroutet} werden)
\end{lstlisting}

\subsection{RFC 1918-Adressen}
%https://datatracker.ietf.org/doc/html/rfc1918
Ein Großteil der IP-Adressen werden \textit{blockweise} an Provider, Firmen und Forschungseinrichtungen herausgegeben und können fortan im Internet genutzt werden. Mit RFC 1918  wurden Blöcke durch die IANA reserviert, welche ausschließliche für private und firmeninterne Nutzung vorgesehen sind. Sie werden im Internet nicht geroutet. Es handelt sich dabei um die Blöcke:
\begin{itemize}
\item 10.0.0.0/8
\item 172.16.0.0/12
\item 192.168.0.0/16
\end{itemize}

Auf diese Blöcke wird auch in der Bachelor-Arbeit zurückgegriffen, um Zugriff auf \textit{interne}, vom Internet \textit{isolierte} Ressourcen zu ermöglichen.

\subsection{IP-Routing}

Wie im vorherigen Beispiel gefolgert, müssen Pakete, die Ziele in anderen Subnetzen haben, geroutet werden. Dafür besitzen alle Netzwerkteilnehmer eine Routing-Tabelle, um herauszufinden, an welchen \glqq Next-Hop\grqq{} ein Paket adressiert werden muss. Im folgenden Beispiel besteht das Netzwerk aus vier Teilnehmern: einem Client, einem Server und zwei Routern (R1 und R2). Alle Geräte verwalten eine Routing-Tabelle, in denen die Zielnetze und Next-Hops vermerkt sind. Damit weiß der Client, dass Pakete für das Netzwerk 192.168.200.0/24 über R1 mit der IP-Adresse 192.168.0.1 geschickt werden müssen. R1 nimmt das Paket an, schaut wiederum in der eigenen Routing-Tabelle nach und schickt das Paket dann weiter an R2, usw. (Rück-)Pakete vom Server zum Client durchlaufen die gleiche Prozedur pro Hop. Client und Server müssen im Übrigen das Netzwerk 192.168.100.0/30 \textbf{nicht} in ihrer Routing-Tabelle pflegen. Dieses Netzwerk hat eine ausschließliche Relevanz zur Übertragung von Daten zwischen R1 und R2: Man spricht auch von Transfernetzwerken.\\

\begin{figure}[h]
  \centering
  \includegraphics{Figures/next_hop_routing_specific_table.pdf}
  \caption{Next Hop Routing}
  \label{grafik: next_hop_routing}
\end{figure}

Dieses Beispiel bringt ein Skalierungsproblem mit sich: Umso mehr Subnetze existieren, in denen Server zu erreichen sind, desto mehr Routen muss der Client in der Routing-Tabelle (manuell) pflegen. Das gleiche gilt für Server, insofern eine Server-zu-Server-Kommunikation gewünscht ist.
In der Praxis sieht man kaum noch Endsysteme (darunter fallen Client und Server), welche mit solchen \textit{spezifischen} Routen arbeiten. Man überlässt die komplexe Verwaltung von Routing-Tabellen den Routern, während die Endsysteme neben der link-lokalen nur noch die \textit{Default}-Route besitzen. In diese Route 0.0.0.0/0 fallen alle Zielnetzwerke, abgesehen vom link-lokalen Subnetz, und sie zeigt immer auf den lokal erreichbaren Router.\\
Das Setzen der Default Route garantiert allerdings nicht, dass ein Ziel auch wirklich erreicht werden kann, z.B. weil es gar nicht existiert oder das Zielnetzwerk nicht in der Routing-Tabelle eines (Next-Hop-)Routers vorhanden ist. Ein Hilfsprotokoll zur Signalisierung von Erreichbarkeit ist dabei das Internet Control Message Protocol (ICMP). Das bekannte Werkzeug Ping nutzt Nachrichten in Form von ICMP Echo Request und ICMP Echo Response, um die Erreichbarkeit eines Endsystems zu prüfen.
\begin{figure}[h]
  \centering
  \includegraphics{Figures/next_hop_routing_default_route.pdf}
  \caption{Next Hop Routing mit Default Route}
  \label{grafik: next_hop_routing_with_default_route}
\end{figure}\FloatBarrier




Prinzipiell spielen im IPv4-Routing viele weitere Mechanismen eine zentrale Rolle: Zu nennen wären an dieser Stelle bspw. MAC-Adressen (Ethernet), Address Resolution Protocol (ARP), Metrik und Time-To-Live (TTL). Für diese Ausarbeitung haben die Themen allerdings keine Relevanz. Das Thema \glqq dynamisches Routing\grqq{} wird später noch behandelt.
%Transfernetzwerk
%keine Metrik / AD
%In den Ursprüngen des Internets waren IP-Adressen \textit{classful}: die Netzwerkgröße wurde über die Klassenzugehörigkeit einer IP-Adresse bestimmt. Ein Class A Netz fiel in die \textit{Range} 0.0.0.0 - 127.255.255.255. So hätte obige IP-Adresse zu dem Class-A-Netzwerk "1.0.0.0" gehört
% \include{Chapters/02_Recherche_und_Auswahl_der_Komponenten}
\chapter{Definition der Use-Cases} \label{Umsetzung der Use-Cases und Evaluation}

\section{Kundenumfeld und typische Bedürfnisse} \label{Kundenumfeld und typische Bedürfnisse}

%kleine und mittlere Deployments... statt KMU. Bachelor-Arbeit war moving target
%Mein Ansatz: bis zu einer technischen Größenordnung ist das okay... Ansonsten Direct Connect oder Express Route
%Für kleine und mittlere Deployments (nicht KMU) -> keine kaufmännische sondern technologische Betrachtung
%Weniger Ebenen!

"Qualitätssicherung": Warum haben sich die Erkenntnisse aus dem Proposal zur Arbeit geändert? Halbe Seit
Wie bereits eingangs erwähnt, sollen die Use-Cases typische Bedürfnisse von Unternehmen aus dem kleinen und mittelständischen Bereich widerspiegeln (KMU). So ist die Grundannahme, dass der Kunde bereits Rechenressourcen in Selbstverwaltung (\glqq Private Cloud\grqq{}) besitzt. Weiterhin hat die Integration von Public Cloud-Diensten noch gar nicht oder nur oberflächlich stattgefunden: Es wurden ein paar Maschinen hochgefahren und bestimmte Internetdienste installiert. Über eine tiefergehende Koppelung mit bestehenden Diensten und Systemen hat der Kunde bisher keine Überlegungen angestellt. %Eine private Adressierung der Systeme ist noch nicht möglich, die Maschinen werden über die öffentlichen IPs aus dem Internet erreicht. Die Firma besitzt eine private Cloud
%ToDo: Genauer beschreiben, was beim Kunden schon zur Verfügung steht (Private Cloud) und dass das abstrahiert (vereinfacht) dargestellt wird


\section{Use Case 1: Basis Deployment}


Es wird ein Basis Deployment benötigt, um eine grundlegende Integration in die Firmeninfrastruktur zu ermöglichen.
%ToDo: Initial erwähnen, dass Amazon Web Services -> AWS, Microsoft Azure -> Azure
Der Fokus dieser Arbeit liegt auf den Public Cloud-Plattformen AWS und Azure sowie einer vereinfacht dargestellten Private Cloud. Die Annahme ist, dass diese bereits zur Verfügung steht und eine Menge an Diensten und Maschinen bereitstellt, die vom Kunden in Selbstverwaltung administriert wird.\\
Eine Hybrid Cloud besteht, wie bereits beschrieben, aus Public und Private Cloud. Da mit beiden Public Clouds getestet werden soll, wäre hier bereits eine Fallunterscheidung notwendig: Private Cloud <-> Azure bzw. Private Cloud <-> AWS. Um sich diese Fallunterscheidung sparen zu können, soll ein Dreieck ausgerollt werden, bei dem jeder Punkt eine Cloud-Plattform darstellt.
%Weitere Vorteile ergeben sich später: Redundanz, einfaches Skalieren _zwischen_ Public Clouds
%Bild Basis Deployment
Optimalerweise wird durch dieses Deployment auch die Redundanz erhöht: Fällt eine Verbindung aus bspw. zwischen AWS und Azure, so können Datenpakete weiterhin über die Private Cloud geroutet werden.
%Bild Link Fail

%Kein NAT -> Ende-zu-Ende-Kommunikation _aller_ Teilnehmer

Der Use-Case soll als Grundaufbau für alle weiteren Use-Cases dienen. Das Dreieck aus AWS, Azure und Private Cloud wird fortan als \textit{Backbone} bezeichnet.

\subsection{Vorauswahl geeigneter technischer Komponenten}
%PFS, AES Best Practices referenzieren
AWS und Azure bieten auf ihren Plattformen Unterstützung für Route-Based IPSEC-Tunnel. Darüber kann eine virtuelle Punkt-zu-Punkt (vgl. technische Grundlagen) hergestellt werden. Gleichzeitig werden übertragene Daten verschlüsselt und dadurch Integrität und Vertraulichkeit geschützt.\\
Weiterhin kann innerhalb der Tunnel BGP gesprochen werden, um ein dynamisches Routing zu ermöglichen \cite[S. 19]{AlShawi2020} \cite[S. 74-79]{Toroman2019}. Das dynamische Routing bietet den Vorteil, dass IPv4-Routen nicht manuell bei allen Gateways des Netzwerks bekannt gemacht werden müssen: Sobald ein Teilnehmer eine neue Route besitzt, wird dies via BGP den restlichen Teilnehmern bekannt gegeben.\\
%In technischen Grundlagen erwähnen: IaC, Provider, Module
%Mozilla Public License v2.0[2]
%Beispiel aus der Realität nennen: Switch - Kabel - Server
Für das automatisierte Deployment der Infrastruktur eignet sich Terraform der Firma Hashicorp. Es besitzt eine Vielzahl an \textit{Resources}, welche es ermöglichen, Infrastrukturkomponenten \texit{reproduzierbar} zu deployen. Im Gegensatz zu anderen Automatisierungsframeworks wie Ansible werden durch die implizite Referenzierung zwischen Ressourcen Infrastruktur-Abhängigkeiten aufgelöst.
Die Reihenfolge von Resource-Aufrufen innerhalb der Terraform-Module spielt keine Reihenfolge. So lassen sich komplexere Infrastrukturen installieren, ohne die Abhängigkeiten manuell auflösen zu müssen. Ressourcen, die keine Abhängigkeiten zueinander haben, werden parallel installiert.

%Todo: RFC 5737 in Einleitung (Documentation IPs)
\begin{lstlisting}[label=terraform-implicit-dependeny,caption=Durch die implizite Referenz auf \textit{aws\_vpc.example\_vpc.id} wird zuerst die Ressource \textit{example\_vpc} ausgeführt]
resource "aws_vpn_gateway" "example_gateway" {
	vpc_id = aws_vpc.example_vpc.id
}

resource "aws_vpc" "example_vpc" {
	cidr_block = "192.0.2.0/24"
	instance_tenancy = "default"
}
\end{lstlisting}

%PHPIPAM: GPLv3, Terraform Provider: Apache License 2.0
%https://web.archive.org/web/20201122173813/https://github.com/lord-kyron/terraform-provider-phpipam
%https://github.com/phpipam/phpipam
%https://web.archive.org/web/20201108102250/https://phpipam.net/
%https://blog.vyos.io/vyos-rolling-release-has-got-an-http-api
%https://web.archive.org/web/20210405210638/https://docs.vyos.io/en/latest/automation/vyos-api.html

Darüber hinaus wird ein IPAM benötigt zur IPv4-Adressverwaltung. Die automatische Zuteilung von Adressbereichen darf nicht dazu führen, dass Adressbereiche mehrfach verteilt werden oder sich Adressbereiche überlappen. Gewählt wurde hier das Werkzeug PHPIPAM: Es lässt sich sehr gut mit Terraform integrieren, da ein entsprechender Provider zur Verfügung steht.\\
Die VPN-Gateways, die das Backbone aufspannen, sind mit \textit{Virtual private gateway} bei AWS und \textit{VPN Gateway} bei Azure gesetzt. Dies sind die typischen \textit{Building-Blocks}, die von den Herstellern für VPN-Verbindungen angeboten werden. Nur wenn sich im Laufe der Arbeit herausstellen sollte, dass diese Systeme nicht interoperabel arbeiten, wird versucht, Alternativen zu finden. Als Router, der die Private Cloud repräsentiert, wurde ein VyOS Router gewählt. Dieser steht als Open Source zur Verfügung, es gibt allerdings auch bezahlten Support für Produktionsumgebungen. Der Router hat IPSEC- und BGP-Unterstützung und besitzt ein Command Line Interface, welches sich für das automatisierte Deployment mit Terraform als sehr vorteilhaft erweist, obwohl bisher kein Terraform Provider zur Verfügung steht (vgl. später). Eine REST-API steht ebenso zur Verfügung, aber diese ist zum Stand der Bachelor-Arbeit \textit{bleeding edge} und bisher eher dürftig dokumentiert. Die Annahme ist, dass der Router und das IPAM in der Private Cloud bereits vorhanden sind. Diese Komponenten sind nicht Teil des (Terraform-)Deployments.\\
Prinzipiell lassen sich viele Router-Modelle nutzen, insofern die Unterstützung für genannte Techniken vorhanden ist, bspw. CSR 1000V der Firma Cisco \cite{Durai2016}. Der offene VyOS Router bietet den Vorteil, dass keine Lizenzen für die Nutzung hinterlegt werden müssen, was in vielen Fällen manuelle Konfigurationen erfordert. Außerdem hätten passende Lizenzen beschafft werden müssen, was u.U. zu Verzögerungen der Bachelor-Arbeit geführt hätte.

\begin{figure}[h]
  \centering
  \includegraphics{Figures/Use-Case-1_Basis_Deployment.pdf}
  \caption{Basis Deployment}
  \label{grafik:Use-Case-1_Basis_Deployment}
\end{figure}


%Kein NAT, da am Anfang evtl. einfacher, aber am Ende nur Ärger und Freischaltungen etc...

%Um den Anspruch der Automatisierung gerecht zu werden, ...
%Evaluationskriterien nummerieren
\subsection{Evaluationskriterien}
Nach der Umsetzung wird evaluiert, ob das Deployment folgender Kriterien erfolgreich war:
\begin{enumerate}
    \item IPv4-Adressbereiche werden im IPAM reserviert und mit AWS VPC und Azure VNET assoziiert. Test-Szenario: Zur Verifizierung werden die reservierten Adressbereiche mit den assoziierten verglichen.
    \item IPSEC-Verbindungen werden zwischen allen VPN-Gateways aufgebaut. Test-Szenario: Dies kann mit einem Blick in die verschiedenen \textit{Dashboards} der Cloud-Plattformen bzw. CLI-Kommando (VyOS) verifiziert werden.
    %FIB in technischen Grundlagen?
    \item BGP-Sessions werden etabliert und Präfixe zwischen den Teilnehmern ausgetauscht. Die Routen müssen in der Routing-Tabelle sichtbar sein. Test-Szenario: Eine Testmaschine pro Site und Ping-Tests zwischen den Sites veranlasst werden.
    %Evtl. noch testen, ob Pings auch bei Verbindungsverlust funktionieren
    \item Die Präfixe sollten im Normalfall über verschiedene AS-Pfade sichtbar sein. Nur bei Verbindungsverlust sind Präfixe ausschließlich über einen AS-Pfad zu sehen. Tests finden analog zu Punkt 2 statt.
\end{enumerate}
\section{Use Case 2: Roadwarrior-VPN}
In diesem Use-Case soll erarbeitet werden, inwiefern sich Mitarbeiter via VPN - im Folgenden Roadwarrior genannt - mit den zuvor ausgerollten Cloud-Infrastrukturen verbinden können. Ralf Spenneberg definiert diesen Begriff wie folgt:\\
\glqq Der Begriff Roadwarrior bezeichnet Personen, die mit unbestimmter IP-Adresse auf ein VPN-Gateway zugreifen wollen. Typischerweise handelt es sich hierbei zum Beispiel um Außendienstmitarbeiter, die von unterwegs Zugriff auf die Datenbanken ihres Mutterunternehmens benötigen. Aber auch alle anderen Konstellation, bei denen Rechner mit dynamischen IP-Adressen eine VPN-Verbindung mit einem VPN-Gateway aufbauen möchten, sind denkbar. Hierbei ist die Anzahl der Roadwarrior nicht beschränkt. Theoretisch und auch praktisch sind mehrere Hundert gleichzeitiger Tunnel möglich.\grqq{} \cite[S. 199]{Spenneberg2010}\\
Gerade die Corona-Krise hat das Ausweichen auf das Home-Office für einige Branchen unverzichtbar gemacht. Einige prophezeien bereits eine neue Arbeitswelt \textit{New Work}, die \glqq flexible Arbeitsgestaltung, zum Beispiel durch Vertrauensarbeitszeit und -orte sowie Verzicht auf standardisierte Kernarbeitszeiten\grqq{} mit sich bringt \cite{Umbs2020}.
Klassischerweise verbindet sich der Roadwarrior mit dem Hauptstandort, um von dort aus weitere interne Ressourcen zu erreichen - in diesem Falle die Private Cloud. Dieses klassische Design bringt insbesondere unter der Annahme, dass sich Dienste in die Cloud verlagern lassen, diverse Probleme mit sich:
\begin{itemize}
\item Der Hauptstandort muss Bandbreite für alle $n$ Roadwarrior zur Verfügung stellen. V.a. zu Beginn der Corona-Krise durften viele Unternehmen und öffentliche Einrichtungen erfahren, dass sie hier zu schwach aufgestellt sind \cite{tufreiberg2021}.
\item Der Hauptstandort ist u.U. weit entfernt. Der Mitarbeiter nimmt auf Grund von hohen Latenzen und eventuellen Paketverlusten eine schlechte Applikations-Performance wahr. Dazu wird oftmals über Smartphone-Hotspot, Hotel-WLAN, o.ä. gearbeitet, welche häufig eine unterdurchschnittliche Internetanbindung anbieten.
%Fulltunnel als Argument?
\item Latenzen erhöhen sich zusätzlich, falls bestimmte Applikations-Server gar nicht am Hauptstandort vorhanden sind, z.B. weil sie in die Public Cloud verlagert wurden.
\end{itemize}
Optimalerweise wird das Design also diese genannten Punkte in Angriff nehmen:
\begin{itemize}
\item Ein Load Balancing der Bandbreiten wird ermöglicht, indem $n$ Roadwarrior über $m$ Clouds verteilt werden.
\item Der Roadwarrior verbindet sich im günstigsten Fall mit dem Standort, der die geringste Entfernung zu ihm aufweist.
\item Häufig genutzte Applikationen sind am jeweiligen Cloud-Standort, mit dem sich der Roadwarrior verbunden hat, verfügbar, um eine gute Usability für den Anwender zu gewährleisten.
\item Weiterhin sollen genannte Maßnahmen für den Anwender möglichst transparent und ohne manuelle Interaktion erfolgen. Er soll keine Wahl haben, mit welchem Cloud-Standort er sich zu verbinden hat: Die Annahme ist, dass dieser über geeignete Automatisierungsmechanismen mit dem \textit{besten} Standort verbunden wird.
\end{itemize}
Das Szenario baut auf Use-Case 1 auf. So wird weiterhin von dem Backbone-Grundaufbau ausgegangen. Es soll gezeigt werden, wie ein Roadwarrior-Setup aufgebaut werden kann, bei dem sich der jeweilige Mitarbeiter mit dem nächstgelegen \textit{Hop} verbindet, um die beste Performance erreichen zu können. Weiterhin soll an jedem Cloud-Standort ein (interner) Server stehen, welcher einen internen Netzwerk-Dienst anbietet. Das Ziel ist es, dass ausschließlich der Server genutzt wird, der an dem Standort zur Verfügung steht. Ansonsten würden sich der Latenzgewinn durch die nahe VPN-Gegenstelle durch die \textit{interne} Latenz wieder aufheben.\\
Die Infrastruktur-Komponenten aus Use-Case 1 werden als vorhanden und funktionierend angenommen und in Abbildung \ref{grafik:Use-Case-2_Vereinfacht} nicht mehr dargestellt. Die Roadwarrior-Clients verbinden sich immer mit dem nächstgelegenen VPN-Konzentrator.
\begin{figure}[h]
  \centering
  \includegraphics[scale=0.75]{Figures/Use-Case_2_Vereinfacht_1.pdf}
  \caption{Use-Case 2: Roadwarrior}
  \label{grafik:Use-Case-2_Vereinfacht}
\end{figure}\FloatBarrier

\subsection{Vorauswahl geeigneter technischer Komponenten}\label{uc1-vorauswahl}
Zur Terminierung der Roadwarrior-Clients muss pro Cloud-Standort ein VPN-Konzentrator zur Verfügung stehen. Mit \textit{Client VPN Endpoint} (AWS) bzw. \textit{Point-to-site} Azure bieten bereits Lösungen, um Roadwarrior-Clients zu terminieren. Nach längerer Evaluation dieser Building Blocks erwiesen sich diese Lösungen für das Ziel der Arbeit als nicht tauglich:
\begin{itemize}
\item Bei AWS werden Client-VPN-Verbindungen eingehend geNATted (\textit{NAT Masquerading}). Damit sind alle Clients intern mit der gleichen Absender-IP zu sehen. Dies verletzt den Anspruch der Arbeit, eine Ende-zu-Ende-Konnektivität zu ermöglichen.
\item Obwohl Azure als auch AWS OpenVPN für Roadwarrior-VPNs benutzen, sind die Client-VPN-Konfigurationen nicht \glqq deckungsgleich\grqq{} zu bekommen. Um den sich mit dem nächstgelegenen VPN-Konzentrator zu verbinden, wäre eine manuelle Interaktion des Benutzers notwendig, was mit den Evaluationskriterien nicht vereinbar ist. Weiterhin kann der Benutzer nicht immer wissen, wo der nächstgelegene Standort ist. Man kann nicht von jedem Mitarbeiter tiefe Kenntnisse der Netzwerktopologie abverlangen.
\end{itemize}
AWS benutzt bspw. eine \textit{remote-random-hostname}-Direktive (s. Listing \ref{ovpn-client-config-remote}), welche dafür sorgt, dass bei Verbindungsversuch eine Zufallszeichenkette an den Domainnamen angehangen wird, um DNS Caching zu verhindern.
%TC:ignore
\begin{listing}[h]
\begin{minted}[breaklines,frame=single]{bash}
$ grep remote *-client-vpn.ovpn
aws-client-vpn.ovpn:remote cvpn-endpoint-08345.prod.clientvpn.eu-central-1.amazonaws.com 443
aws-client-vpn.ovpn:remote-random-hostname
aws-client-vpn.ovpn:remote-cert-tls server
azure-client-vpn.ovpn:remote azuregateway-61820068.vpn.azure.com 443
azure-client-vpn.ovpn:remote-cert-tls server

\end{minted}
\caption{Auszüge aus den OpenVPN-Client-Konfigurationen für AWS und Azure.}
\label{ovpn-client-config-remote}
\end{listing}\FloatBarrier
%TC:endignore
Weiterhin sind die unterschiedlichen Authentifizierungsmechanismen nur schwierig miteinander zu kombinieren. Auch hier wäre eine Benutzerinteraktion notwendig.\\
Daher wurde für weitere Überlegungen von den offiziellen Building-Blocks abgesehen und entschieden, pro Standort einen eigenen OpenVPN-Server hochzufahren: OpenVPN ist freie Software und man muss sich daher nicht mit Lizenzierungen beschäftigen analog zu VyOS.\\
Aus den Erfahrungen aus Use-Case 1 wurde dafür ebenso die Router-Distribution VyOS gewählt. Diese kann nicht nur für Site-to-Site-Tunnel genutzt werden, sondern kann auch für Client-to-Site-Tunnel terminieren, um den Zugang für Roadwarrior zu ermöglichen.\\
Denkbar wäre ebenso eine simple Linux-Distribution mit dem Paket OpenVPN: VyOS bietet den Vorteil, dass sich die notwendigen Konfigurationen ebenso über CLI erledigen lassen und somit per Terraform-Template \textit{automatisiert} ausgebracht werden können\cite{vyosopenvpn2021}. VyOS ist im AWS als auch im Azure Marketplace zu beziehen.\\
Ein Problem ergibt sich aus der Authentifizierung und Autorisierung der Roadwarrior-Clients. Wenn Protokolle wie RADIUS\cite{rfc2865} als Authentifizierungsmechanismus mit Benutzerkennung und Passwort, muss man entweder
\begin{enumerate}[label=(\alph*)]
\item \label{radius-decentralized} einen zentralen RADIUS-Server pro Standort verwalten oder
\item \label{radius-centralized} einen zentralen RADIUS-Server für alle Standorte zur Verfügung stellen.
\end{enumerate}
\ref{radius-decentralized} bringt ein Skalierungsproblem mit sich: So müssen bspw. Benutzerkennungen und Passwörter lokal auf allen Systemen gepflegt werden, was die Komplexität des Use-Cases deutlich erhöhen würde und auch in Realität schwierig zu warten wäre. Mit \ref{radius-centralized} hätte man einen Single-Point-of-Failure, scheidet also auch aus. Hätte der \textit{zentrale} Authentifizierungsserver ein technisches Problem, wären alle VPN-Konzentratoren betroffen und Roadwarrior könnten sich nicht mehr einwählen.\\
Lightweight Directory Access Protocol (LDAP)\cite{rfc4511} als verteilte Datenbank wäre eine Lösung für genannte Probleme: man müsste pro Standort einen LDAP-Server bereitstellen und durchgehend mit allen weiteren LDAP-Servern synchronisieren. Als Authentifizierungsmechanismus könnte man dann bspw. RADIUS\cite{rfc2865} oder gleich LDAP benutzen. \textit{Windows Domänen Controller} implementieren den LDAP-Verzeichnisdienst in Form von \textit{Active Directory} und wären ein Beispiel für solch ein verteiltes Szenario.\cite[S.603-604]{Tanenbaum2003}\\
In dieser Arbeit wird von der Implementation abgesehen, da die Komplexität deutlich erhöht und es für den Proof-of-Concept an sich keine gewinnbringenden Erkenntnisse liefern würde.\\
Der Lösungsansatz wäre an dieser Stelle, auf eine Zertifikat-basierte Authentifizierung zu setzen. Eine Publik-Key-Infrastruktur benötigt zur Authentifizierung keine Verbindungen zu anderen (Authentifizierungs-)Services. Es reicht, auf dem VPN-Server vertrauenswürdige Root-Zertifikate zu hinterlegen. Diese \textit{vertrauenswürdigen Stellen} signieren mit dem privaten Schlüssel dann die Zertifikate der Roadwarrior-Clients, wodurch das Vertrauensverhältnis zwischen VPN-Server und -Client hergestellt ist.\\ 
Auf Prüfung der Gültigkeit von Zertifikaten mittels Certificate Revocation Lists (CRLs) bzw. Online Certificate Status Protocol (OCSP) wird in dieser Arbeit verzichtet. Sie ist aber prinzipiell möglich mit der serverseitigen OpenVPN-Direktive \textit{crl-verify} (CRL-Prüfung) bzw. \textit{tls-verify} (OCSP-Prüfung).\cite[S.116, S.325-327]{Keijser2011}\\
%tf remote
Als PKI wird eine pfSense-Instanz genutzt. Dieses Firewall-Betriebssystem bietet eine simple PKI, welche via Web-GUI administrierbar ist und so das Ausstellen und die Verwaltung der Client- und Server-Zertifikate erleichtert.\cite[S.376-383]{Netgate2020}

%Evaluationskriterien nummerieren
Der technische Fokus wird in diesem Use-Case auf DNS liegen. Es soll in dem Szenario zwei essenzielle Aufgaben lösen, die durch die Evaluationskriterien vorgegeben werden:
\begin{itemize}
\item Mit Hilfe von GeoIP-Daten soll ermöglicht werden, immer den nächstgelegenen VPN-Konzentrator zu finden. Wenn kein Server in der Nähe ist, soll ein beliebiger Server kontaktiert werden (random)\cite{bindrrset2020}. Wenn das Terraform \gls{Deployment} noch nicht stattgefunden hat, also keine Public Cloud-Standorte zur Verfügung stehen, soll sich der Client immer mit dem VPN-Konzentrator der Private Cloud verbinden.
\item Sobald der Roadwarrior mit einem VPN-Konzentrator verbunden ist, soll immer der (beispielhafte) Server, der am jeweiligen Standort bereitgestellt wurde, kontaktiert werden.
\end{itemize}

Mit Hilfe des Bind Nameservers ab Version 9.10 ist es möglich, die GeoIP-Datenbank des Anbieters MaxMind auszuwerten, um DNS-Anfragen in Abhängigkeit der Absender-IP zu beantworten\cite{bindgeoip2020}. Bei dieser IP handelt es sich typischerweise um die Resolver-IP des \textit{lokalen} Internet-Dienstleisters. Die Annahme für diesen Use-Case ist daher, dass der DNS-Resolver in der gleichen Region wie der Roadwarrior angesiedelt ist\label{dns-resolver-region}.\\
Um mit GeoIP-Daten der Firma MaxMind arbeiten zu können, muss Bind mit einem speziellen Flag (\texttt{-{}-with-maxminddb}) kompiliert werden. Im Ubuntu 20.04 Standardpaket von Bind 9 ist dieses Feature bereits vorhanden (s. Listing \ref{bind-mmdb-compiler-flag}).

%TC:ignore
\begin{listing}[h]
\begin{minted}[breaklines,frame=single]{bash}

$ named -V | grep -o -- '--with-maxminddb'
--with-maxminddb

\end{minted}
\caption{Das Ubuntu 20.04 Standardpaket wurde bereits mit dem Flag kompiliert.}
\label{bind-mmdb-compiler-flag}
\end{listing}
%TC:endignore
Der Nameserver muss sowohl extern, also aus dem Internet, als auch intern für verbundene VPN-Clients, erreichbar sein. Der Nameserver wird ausschließlich in der Private Cloud gehostet. Es wird von einem \gls{Deployment} eines \textit{secondary} Name-Servers \textit{pro} Standort zur Vereinfachung des \gls{Deployment}s abgesehen. Es sind keine Erkenntnisgewinne dadurch gegeben. Secondary Name-Server halten Kopien aller Zonen des primary Name-Servers vor, welcher für die Verwaltung von Zonendateien zuständig ist \cite[S.517]{Fall2011}.\\
In der Praxis sollte das \gls{Deployment} von \glqq Secondaries\grqq{} aus Redundanzgründen jedoch dringend in Betracht gezogen werden. Außerdem können \glqq verzögerte\grqq{} DNS-Antworten auf Grund sehr großer Entfernungen negative Auswirkungen auf die Usability haben. Da alle Test-Standorte innerhalb von Europa liegen, ist dieser Effekt im Sinne des Proof-of-Concepts vernachlässigbar.\\
Weiterhin müssen Zonen dynamisch editierbar sein, da durch das Terraform-\gls{Deployment} virtuelle Maschinen mit vorher unbekannten IP-Adressen erstellt werden. Diese IP-Adressen müssen mit den Domains der jeweiligen Zonen kombiniert werden. Dies gilt für extern und intern auflösbare Zonen. Für Terraform steht ein Provider \textit{dns} zur Verfügung, der diese dynamischen Updates erlaubt\cite{dnstf2021}.
Das in diesem Kapitel angedeutete Szenario soll wieder per Terraform bereitgestellt werden und setzt direkt auf Use-Case 1 auf. Auf Grund der langen Dauer des \gls{Deployment}s in Use-Case 1 (siehe Kap. \ref{azure-deployment-time}) wird erwogen, Use-Case 2 nicht \glqq komplett\grqq{} auszurollen, sondern Use-Case 1 zu belassen und die Infrastruktur-Komponenten aus Use-Case 2 lediglich hinzuzufügen (vgl. Kap. \ref{accelerate-deployment-use-case-2}).\\
Die OpenVPN-Server sollen auflösbar sein unter der Domain vpn.ba.mungard.de.\\
Als stellvertretender \glqq Applikations-Server\grqq{} soll ein simpler Apache2 pro Standort in einer Ubuntu-VM installiert werden. Dieser soll über die Domain www.intern.ba.mungard.de erreichbar sein.
%Simulation Applikation-Server mit Apache2 HTTP Server
%Authentisierung
%GeoIP vs. Anycast, bspw. Google DNS 8.8.8.8 -> warum ist das hier nicht praktibel (Kunden haben keine eigenen Netze und peeren nicht weltweit)
%
%AWS == Frankfurt, Azure == Dublin, Kiel
%CRL / OCSP
\subsection{Evaluationskriterien}\label{eval-roadwarrior}
\begin{enumerate}
\item Roadwarrior-Clients werden simuliert, indem an den entsprechenden Standort (Kiel, Dublin bzw. Frankfurt) eine weitere virtuelle Maschine installiert wird, die VPN-Verbindungen zu den jeweiligen VPN-Konzentratoren aufbaut.
\item Ein Roadwarrior-Client verbindet sich immer mit dem nächstgelegen Standort. Die DNS-Antworten werden per \texttt{dig}- und \texttt{whois}-Kommando verifiziert.
\item Der Client hat keine Wahl, mit welchem Standort dieser verbunden wird.
\item Ist ein Cloud-Standort nicht verfügbar (z.B. weil defekt oder nicht bereitgestellt), \textit{muss} der Client auf den \textit{default} Standort (Kiel) zurückfallen.
\item Sobald ein Client mit einem VPN verbunden ist, muss er die entsprechende interne Server-Ressource ansprechen, die am jeweiligen Standort zur Verfügung steht.
\end{enumerate}

Weitere Evaluationskriterien sind in der Praxis denkbar, spielen im Kontext \textit{Proof-of-Concept}) jedoch keine Rolle.
\chapter{Umsetzung der Use-Cases und Evaluation} \label{Umsetzung der Use-Cases und Evaluation}

\section{Use-Case 1: Basis Deployment} \label{Use-Case 1: Basis Deployment}
%IPAM-Vorbereitungen, TF Matching erläutern
%Was ist ein Subnet im Cloud-Kontext? Einleitung?
%TF: Resource, Data, Modul, Provider
\subsection{Umsetzung: Kerntätigkeiten}

Im PHPIPAM müssen mehrere Netzbereiche reserviert werden. Es werden Netzbereiche benötigt, in denen Maschinen per IPv4 kommunizieren können. Diese Netze werden mit VPC bzw. VNET assoziiert. Für VNET wurde der Netzblock \glqq 10.32.0.0/16\grqq{} und für VPC der Netzblock \glqq 10.33.0.0/16\grqq{} vorreserviert, aus dem kleinere {Subnets} für die jeweilige Cloud entnommen werden können. Weiterhin ist die Annahme, dass die Private Cloud bereits einen Netzbereich besitzt, in dem Maschinen angesiedelt sind: \glqq 192.168.201.0/24.\grgg{}\\
%https://docs.aws.amazon.com/vpn/latest/s2svpn/VPNTunnels.html
%https://docs.microsoft.com/de-de/azure/vpn-gateway/bgp-howto
Weiterhin benötigt man {Transfernetzwerke}, über die die IPv4-Pakete geschickt werden und die BGP-Präfixe ausgetauscht werden können. AWS sieht hierfür /30-Präfixe aus dem Bereich \glqq 169.254.0.0/16\grqq{} vor, Azure hat eine Range reserviert: \glqq 169.254.21.0 - 169.254.22.255\grqq{}. Als Kompromiss können daher nur Netze aus den Azure-Bereichen genutzt werden, da die Bereiche, die AWS zur Verfügung stellt, größtenteils außerhalb dieser Range liegen. Auch diese Netzbereiche müssen im IPAM vorreserviert werden, auch die Transfernetze automatisiert ausgebracht werden.\\
Im folgenden Code-Beispiel wird innerhalb des IPAM nach dem Bereich \glqq TF\_CLOUD\_BGP\_TRANSFER\grqq{}, aus dem alle Transfernetze entommen werden, gesucht. Innerhalb des Bereichs wird aus dem vorreservierten Block \glqq Cloud\_Transfer\_BGP\_1\grqq{} ein /30-Netzwerk entnommen. Die Reservierung der Netzblöcke für VPC und VNET erfolgen analog.
%data vs. resource in Einleitung erläutern?
\FloatBarrier
\begin{lstlisting}[float,label=network-reservation-ip,caption=Die data-Anweisungen dienen ausschließlich der Suche nach dem passenden Transfernetzwerk-Block. Per resource-Anweisung wird ein /30-Netzwerk reserviert.]
//Azure - AWS Transfer
data "phpipam_section" "apipa_main_section" {
  name = "TF_CLOUD_BGP_TRANSFER"
}
data "phpipam_subnet" "apipa_transfer_subnet" {
  section_id = data.phpipam_section.apipa_main_section.section_id
  description_match = "Cloud_Transfer_BGP_1"
} 
resource "phpipam_first_free_subnet" "free_subnet_apipa" {
  parent_subnet_id = data.phpipam_subnet.apipa_transfer_subnet.subnet_id
  subnet_mask = 30
  description = "BGP_AWS_AZURE"
}
\end{lstlisting}
\FloatBarrier
Es ist auch möglich, dass IPv4-Adressen für Transfernetze automatisch von Azure bzw. AWS vorgegeben werden. Mit dem avisierten Backbone-Design aus drei Teilnehmern lässt sich das allerdings nicht vereinbaren: die Adressbereiche müssen vom IPAM vorgegeben werden (vgl. später: Probleme statische Route).\\
Weiterhin werden Pre-Shared-Keys für die Aushandlung der IPSEC-Verbindungen benötigt. AWS erzeugt automatisch solche Schlüssel, welche durch das Terraform-Modul \glqq aws\_vpn\_connection\grqq{} zurückgegeben werden. Diese automatisch erzeugten Schlüssel werden daher für die Backbone-Verbindungen AWS <-> Azure und AWS <-> Private Cloud (VyOS) genutzt. Da Azure diese Schlüssel nicht erzeugt, wurde ein Terraform-Modul geschrieben, welches diese Schlüssel aus 24 Zeichen erzeugt. Dieses kann anschließend genutzt werden für die Backbone-Verbindung Azure <-> Private Cloud (VyOS).

\begin{lstlisting}[label=tf-generate-psk,caption=Auf Sonderzeichen (special) wurde verzichtet. Das Modul stellt per output-Anweisung den Schlüssel für andere Terraform-Module zur Verfügung.]
resource "random_password" "random_password" {
  length = 24
  lower = true
  upper = true
  number = true
  special = false
}
output "azure_vyos_tunnel1_psk" {
  value = random_password.random_password.result
  sensitive = true
}
\end{lstlisting}

%alles Devices nennen?
Da, wie bereits erwähnt (vgl. klick), für die VyOS-Router keine Terraform-Integration verfügbar ist, wurde zur Konfiguration des Devices eine Vorlage erstellt.

%https://www.terraform.io/docs/language/functions/templatefile.html
Die Erzeugung der Vorlage erfolgt über die Funktion templatefile().

\begin{lstlisting}[label=tf-call-tpl-generation,caption=Die Funktion templatefile() erzeugt aus einer Template-Datei (vyos\_config.tpl) die Datei /tmp/example.sh. Die Template-Datei wird mit den Variablen var.aws\_vgw\_ip und aws\_tunnel1\_psk befüllt.]
resource "local_file" "vyos_config" {
  filename = /tmp/example.sh
  content = templatefile("${path.module}/include/vyos_config.tpl",
  {
    aws_vgw_ip = var.aws_vgw_ip
    aws_tunnel1_psk = var.aws_tunnel1_psk
    [ ... weitere Aufrufparameter ...]
  })
}
\end{lstlisting}

\begin{lstlisting}[label=tf-generate-tpl,caption=Verschiede set-Kommandos werden in ein VyOS-Skript eingebettet (Interpreter: /bin/vbash). Die Variablen in Zeilen 9-12 resultieren aus dem Funktionsaufruf (s.o.).]
$ head -n 11 < vyos/include/vyos_config.tpl
#!/bin/vbash
source /opt/vyatta/etc/functions/script-template
configure
#AWS
set vpn ipsec ike-group AWS lifetime '28800'
set vpn ipsec ike-group AWS proposal 1 dh-group '2'
set vpn ipsec ike-group AWS proposal 1 encryption 'aes128'
set vpn ipsec ike-group AWS proposal 1 hash 'sha1'
set vpn ipsec site-to-site peer ${aws_vgw_ip} authentication mode 'pre-shared-secret'
set vpn ipsec site-to-site peer ${aws_vgw_ip} authentication pre-shared-secret ${aws_tunnel1_psk}
set vpn ipsec site-to-site peer ${aws_vgw_ip} description 'VPC tunnel 1'
\end{lstlisting}

\begin{lstlisting}[label=tf-generate-psk,caption=Das so generierte Shell-Skript wird per SSH auf das Zielsystem (VyOS-Router) hochgeladen und per provisioner remote-exec ausgeführt.]
resource "null_resource" "vyos_config" {
  connection {
    type = "ssh"
    host = var.vyos_host
    user = var.ssh_user
    private_key = file(var.private_key_file)
  }
  provisioner "file" {
   source = /tmp/example.sh
   destination = var.vyos_script_path
  }
  provisioner "remote-exec" {
    inline = [ "chmod +x ${var.vyos_script_path}", var.vyos_script_path ]
  }
}
\end{lstlisting}

\newpage
\subsection{Probleme und Lösungsfindung}
%Probleme: Race Condition externe IP, Redeploy wegen TF Bug, statische Route zu BGP Neighbor (Azure - AWS: Pseudo directly, Azure - VyOS: Static, Azure - AWS), stateless VyOS Config, lange Laufzeiten Erstellung VPN Gateway Azure

\textbf{\underline{Vertauschung von IPSEC-Tunnelparametern}}

%https://github.com/hashicorp/terraform-provider-aws/issues/396
AWS VPN-Verbindungen bieten aus Redundanzgründen standardmäßig zwei IPs pro Gegenstelle an. 

\begin{lstlisting}[label=tf-xml-response-aws,caption=Die ursprüngliche (gekürzte) XML-Antwort der AWS API, welche in der Datei terraform.tfstate gespeichert wurde.]
<?xml version="1.0" encoding="UTF-8"?>
<vpn_connection id="vpn-096f7fc91dc77cc74">
  [...]
  <ipsec_tunnel>
    <vpn_gateway>
      <tunnel_outside_address>
        <ip_address>18.194.163.131</ip_address>
      </tunnel_outside_address>
      [...]
    </vpn_gateway>
    <ike>
      [...]
      <pre_shared_key>B28VDw7xcqYJZvcWCI7EKfaSt6KXe_HC</pre_shared_key>
    </ike>
    [...]
  </ipsec_tunnel>
  <ipsec_tunnel>
    [...]
    <vpn_gateway>
      <tunnel_outside_address>
        <ip_address>35.157.252.106</ip_address>
      </tunnel_outside_address>
    [...]
    <ike>
      [...]
      <pre_shared_key>cHgnL5wZ1vPGTdHhbuLmsCzyC22rcQAW</pre_shared_key>
    </ike>
    [...]
  </ipsec_tunnel>
</vpn_connection>

\end{lstlisting}

Wie man sieht, taucht das XML-Element ipsec\_tunnel zwei Mal auf, ist dabei jedoch unnummeriert.
%https://registry.terraform.io/providers/hashicorp/aws/latest/docs/resources/vpn_connection#attributes-reference
Der Terraform-Provider aws\_vpn\_connection nutzt diese XML-Antwort, um daraus die Rückgabewerte tunnel1\_* und tunnel2\_* zu erzeugen, die in der weiteren Ausführung von anderen Modulen wiederverwendet werden. Bei einigen Ausführungen wurden diese Elemente Terraform-intern vertauscht: Da nun weder IPSEC-Schlüssel noch interne BGP-Peering-Gegenstellen stimmen, kommen weder IPSEC-Tunnel noch die BGP-Peering-Sessions hoch.

%https://docs.aws.amazon.com/AWSEC2/latest/APIReference/throttling.html
Mit einem Bash-Skript konnte zumindest ein Workaround gefunden werden, um ein erfolgreiches Deployment sicherzustellen. Die Grundannahme ist, dass pro Präfix zwei Pfade auf dem Private Cloud-Router existieren. Wenn nur ein Pfad pro Präfix zur Verfügung steht, kam es mit einer hohen Wahrscheinlichkeit zu einer Vertauschung und es wird redeployed (destroy -> apply). Es wird maximal drei Mal versucht, ein funktionierendes Deployment hochzuziehen, um die APIs der Cloud-Anbieter zu \texit{schonen} und kein Rate-Limiting zu triggern.

\begin{lstlisting}[label=tf-base-deployment-bgp-ok,caption=BGP Status]
vyos@vyos-cloud:~$ sh ip bgp | grep -A1 -E '10.3[2|3]'
*  10.32.0.0/16     169.254.53.1           100             0 65516 65515 i
*>                  169.254.21.6                           0 65515 i
*  10.33.0.0/24     169.254.21.6                           0 65515 65516 i
*>                  169.254.53.1           100             0 65516 i
\end{lstlisting}

\begin{figure}[h]
  \centering
  \includegraphics{Figures/programmablaufplan_bash_deploy_tf.pdf}
  \caption{TF Redeployment Bash Skript}
  \label{grafik:programmablaufplan_bash_deploy_tf}
\end{figure}

%Programmablaufdiagramm

Durch das Bash-Skript wurde dieser Bug nicht zum Showstopper, allerdings hat das Deployment manchmal länger gedauert (Normale Deployment Zeit * Versuche).

%FIX vom https://github.com/hashicorp/terraform-provider-aws/pull/19077
Dieser Bug wurde angeblich noch kurz vor Abgabe dieser Arbeit mit PR# gefixt, allerdings konnte das nicht mehr getestet werden. Es ist im Commit ersichtlich, dass es sich um einen Bug in der Sortierfunktion handelte.

\textbf{\underline{Zustandslose VyOS-Konfiguration}}\\
%Terraform Einleitung
Terraform speichert, wie bereits erläutert, alle Referenzen auf Infrastrukturkomponenten in der Datei terraform.tfstate. Da es für VyOS keinen vollwertigen Terraform-Provider gibt, wurde mit Templates, die verschiedene set-Kommandos ausführen, gearbeitet. Diese Konfigurationsänderungen sind dadurch völlig zustandlos, da sie nicht invertierbar sind: Terraform kennt nicht die Umkehrung der "set"-Kommandos, welche benötigt würden, um bei einen \glqq terraform destroy\grqq{} die Konfigurationen am VyOS-Router rückgängig zu machen.
%https://www.terraform.io/docs/language/resources/provisioners/syntax.html#destroy-time-provisioners
Ein Ansatz wäre gewesen, ein weiteres Template zur Verfügung zu stellen, in dem alle set- durch delete-Kommandos negiert werden.\\
\begin{lstlisting}[label=set-delete-vyos,caption=delete negiert das vorherige set-Kommando.]
vyos@vyos-cloud# set system time-zone Europe/Berlin
[edit]
vyos@vyos-cloud# delete system time-zone Europe/Berlin
[edit]
\end{lstlisting}
Per \texit{Destroy-Time Provisioner} würde dieses Template nur bei einem \texit{terraform destroy} genutzt werden. Das Problem ist, dass schon bei minimalen Änderungen des Deploy-Templates auch das Destroy-Template angepasst werden muss. Weiterhin sind die IPSEC-Konfigurationen relativ umfangreich. Es war zu befürchten, dass Relikte aus der Konfiguration übrig bleiben, wenn die Negation aus unbekannten Gründen fehlschlug.

Eine weitere Idee war, mit dem VyOS-Feature "rollback" zu arbeiten. Über eine commit-History werden alle Änderungen an dem System dokumentiert.
\begin{lstlisting}[label=commit-history-vyos,caption=Commit History VyOS]
vyos@vyos-cloud:~$ show system commit | head -3
0   2021-05-05 09:47:51 by tf via cli
1   2021-05-05 09:25:24 by tf via other
2   2021-05-04 21:17:57 by vyos via cli
\end{lstlisting}

So würde man die letzte Revision \textit{vor} der Konfiguration der Backbone-Verbindungen festhalten, um bei einem \texit{terraform destroy} darauf zurückgehen zu können.

\begin{lstlisting}[label=rollback-cmd-vyos,caption=Rollback auf Revision N nach Reboot]
vyos@vyos-cloud# rollback
Possible completions:
  <N>   Rollback to revision N (currently requires reboot)

  Revisions:
    0   2021-05-05 09:47:51 tf by cli
    1   2021-05-05 09:25:24 tf by other
    2   2021-05-04 21:17:57 vyos by cli
\end{lstlisting}

Es wurden zwei Bash-Skripte geschrieben: \texit{save\_last\_manual\_commit.sh} speichert die letzte Revision lokal auf dem VyOS-Router, \texit{apply\_last\_manual\_commit.sh} macht einen Rollback.

\begin{lstlisting}[label=save-last-commit-vyos,caption=Bla]
resource "null_resource" "vyos_config" {
  connection {
    type = "ssh"
    host = var.vyos_host
    user = var.ssh_user
    private_key = file(var.private_key_file)
  }
  [...]
  provisioner "local-exec" {
    command = "${path.module}/bin/save_last_manual_commit.sh"
  }
}
\end{lstlisting}

Nach \texit{terraform apply} liegt auf dem VyOS-Router eine Datei, die den letzten Timestamp der letzten \glqq händischen\grqq{} festhält.

\begin{lstlisting}[label=save-last-commit-file-vyos,caption=Blub]
vyos@vyos-cloud# cat ~tf/last_manual_commit.txt
2021-05-05 09:47:51 by tf via cli
\end{lstlisting}

Bei \texit{terraform destroy} wird das Skript \texit{apply\_last\_manual\_commit.sh} via Terraform \texit{Destroy-Time Provisioner} ausgeführt, welches den Rollback hin zu dem gespeicherten Zeitpunkt auf dem VyOS Router veranlasst.

\begin{lstlisting}[label=apply-last-commit-vyos,caption=Blub]
resource "null_resource" "vyos_config_destroy" {
  provisioner "local-exec" {
    when = destroy
    command = "${path.module}/bin/apply_last_manual_commit.sh"
  }
}
\end{lstlisting}

\textbf{\underline{Lange Laufzeiten für Erstellung von Azure VPN Gateway}}\\
%https://docs.microsoft.com/en-us/azure/vpn-gateway/vpn-gateway-about-vpngateways#whatis
Die Erstellung des Azure VPN Gateways kann bis zu 45 Minuten in Anspruch nehmen. In der Praxis waren es meist ~25 Minuten für den Standort "North Europe". Auch das Löschen eines VPN Gateways via \textit{terraform destroy} dauerte bis zu 15 Minuten. Nur wenn dieses Gateway gelöscht wurde, kann ein \textit{Re-Deploy} erfolgen, welches vor allem dann notwendig ist, wenn die Vertauschung von IPSEC-Parametern (s.o.) eintrifft.\\
Bei diesem Phänomen handelt es sich nicht um einen Bug im eigentlichen Sinne, macht aber das Deployment z.B. für Live-Demonstrationen nicht praktikabel. Da alle weiteren Use-Cases auf diesem Use-Case 1 aufbauen und ebenso in Terraform abgebildet werden sollten, musste eine Lösung gefunden werden, um die Deployment-Zeiten zu verkürzen, da jede Änderung des Codes ein (teilweises) Re-Deployment notwendig macht (vgl. später).

%Dies wird zum Beispiel gebraucht, wenn an der Konfiguration der Infrastruktur Änderungen vorgenommen werden, da Terraform durch diese State-Datei weiß, dass 

\iffalse
%Deadlock? Race Condition?
\textbf{\underline{Deadlock Site-2-Site-Verbindung AWS <-> Azure}}
%Fakeconnection erzeugen

Beim Testen der ersten geschriebenen Terraform-Module kam es zu einer Deadlock-Situation, bei der das \glqq terraform apply\grqq{}-Kommando in einen Timeout lief.
\begin{lstlisting}[label=timeout_aws_azure_ip_deadlock,caption=Deadlock AWS <-> Azure.]
BLA
\end{lstlisting}

%Abgrenzung: alle Public-IP und Private-IP beziehen sich auf IPv4!!!
Bei genauerer Untersuchung stellte sich heraus, dass kein Bug in Terraform vorlag, vielmehr hatte sich ein Logikbug eingeschlichen. Azure erzeugt standardmäßig die Public-IP-Adresse für das Azure-eigene VPN Gateway erst, sobald die \textit{erste VPN-Gegenstelle} konfiguriert wurde. Für die Konfiguration muss allerdings die Public-IP der jeweiligen Gegenstelle angegeben werden. Das Virtual private gateway von AWS legt eine Public-IPv4-Adresse pro Gegenstelle an: 

Dieses Problem ist nicht Terraform geschuldet, sondern eher der Logik, wie die VPN-Verbindungen in Azure bzw. AWS konfiguriert werden 
Dieses Problem tritt in dieser Form nicht mit Verbindungen mit dem Private Cloud-Router auf, da die Public IP-Adresse bereits vor der Ausführung von Terraform bekannt ist.
Fake Gegenstelle...
\fi
\subsection{Evaluation}
Nach dem erfolgreichen Deployment wird folgender Status von Terraform zurückgemeldet. Die letzte Zeile stammt vom Deploy-Skript:
\begin{lstlisting}[label=tf-base-deployment-ok,caption=Terraform Deployment Status]
Apply complete! Resources: 29 added, 0 changed, 0 destroyed.

Outputs:

aws_subnet_id = "subnet-0192ae38a3a16f584"
[... weitere Output-Variablen ...]
\end{lstlisting}

Die im IPAM reservierten Netze sind in AWS als Subnets wiederzufinden. Die Prüfung analog für Azure ist ebenso erfolgreich.

%AWS Screenshot
%IPAM Screenshot


\begin{lstlisting}[label=tf-base-deployment-ipsec-ok,caption=IPSEC Status]
vyos@vyos-cloud:~$ show vpn ipsec sa
Connection                     State    Uptime
-----------------------------  -------  --------
[...]
peer-20.67.209.254-tunnel-vti  up       4h43m19s [...]
peer-3.65.181.187-tunnel-vti   up       4h47m26s [...]
\end{lstlisting}

Die Adressen gehören dabei zu Amazon bzw. Microsoft.

\begin{lstlisting}[label=whois-amazon-public-ip,caption=Diese Adresse gehört Amazon. Analog wurde geprüft für 20.67.209.254 (Microsoft).]
$ whois 3.65.181.187 | grep -A2 NetRange
NetRange:       3.64.0.0 - 3.79.255.255
CIDR:           3.64.0.0/12
NetName:        AMAZON-FRA
\end{lstlisting}

Außerdem wurden auf dem Cloud-Router die AWS- und Azure-Präfixe via BGP installiert. Pro Präfix sind zwei Pfade vorhanden (s.o.).
...s. Listing xy in Lösungsfindung...

Es wurde pro Cloud eine virtuelle Maschine (Ubuntu) installiert, um die Ende-zu-Ende-Konnektivität zu prüfen. Dieser Ping von AWS VPC -> Azure VNET war erfolgreich. Analog waren alle weiteren Ping-Tests zwischen den Cloud-Plattformen erfolgreich:

\begin{lstlisting}[label=tf-base-deployment-ping-ok,caption=Ping Tests zwischen verschiedenen Cloud-Plattformen]
ubuntu@ip-10-33-0-121:~$ ping -c1 10.32.0.4
PING 10.32.0.4 (10.32.0.4) 56(84) bytes of data.
64 bytes from 10.32.0.4: icmp_seq=1 ttl=63 time=25.7 ms

--- 10.32.0.4 ping statistics ---
1 packets transmitted, 1 received, 0% packet loss, time 0ms
rtt min/avg/max/mdev = 25.729/25.729/25.729/0.000 ms
\end{lstlisting}

%Härtung IPSEC Parameter? GCM usw...
%BFD für schneller Konvergenzzeiten? -> Ausblick...
%RTT und Ping in Einleitung / ICMP

Es wurden außerdem in von allen Teilnehmern zu allen Teilnehmern Ping-Tests gemacht und währenddessen eine Verbindung getrennt. Da ein Präfix über zwei Pfade zu sehen ist, ist die Erwartung, dass ein Backup-Pfad genutzt wird. Man kann an folgendem Beispiel sehr gut erkennen, dass das BGP eine Konvergenzzeit erfordert, bis die Pakete über einen anderen Pfad laufen. \texit{ping} versendet in der Standardkonfiguration im Interval von einer Sekunde ICMP-Echo-Pakete. Daraus lässt sich schlussfolgern, dass die Verbindung im unteren Beispiel etwa 44 Sekunden unterbrochen war. \\
Weiterhin hat sich die Round-Trip-Time von ~20 ms auf ~62 ms erhöhtt. Dies ist eine logische Konsequenz, da Pakete nun längere Wege zurücklegen müssen. Alle Ping-Tests waren trotz Kappen einer Verbindung erfolgreich. Man muss dabei allerdings Paketverluste während der Konvergenzzeit hinnehmen.
%Typische BGP Konvergenzzeiten?

\begin{figure}[h]
  \centering
  \includegraphics{Figures/Use-Case-1_Basis_Deployment_missing_link.pdf}
  \caption{TF Deployment with missing link}
  \label{grafik:Use-Case-1_Basis_Deployment_missing_link}
\end{figure}

\begin{lstlisting}[label=tf-base-deployment-ping-ok,caption=Ping Tests zwischen verschiedenen Cloud-Plattformen mit Kappen einer Backbone-Verbindung.]
root@www:~# ping 10.33.0.121
PING 10.33.0.121 (10.33.0.121) 56(84) bytes of data.
[...]
64 bytes from 10.33.0.121: icmp_seq=15 ttl=61 time=19.9 ms
64 bytes from 10.33.0.121: icmp_seq=16 ttl=61 time=24.3 ms
64 bytes from 10.33.0.121: icmp_seq=17 ttl=61 time=19.6 ms
64 bytes from 10.33.0.121: icmp_seq=18 ttl=61 time=19.7 ms <-- Kappen einer Verbindung
64 bytes from 10.33.0.121: icmp_seq=62 ttl=61 time=67.4 ms <-- Revovery über Backup-Pfad
64 bytes from 10.33.0.121: icmp_seq=63 ttl=61 time=62.2 ms
64 bytes from 10.33.0.121: icmp_seq=64 ttl=61 time=62.5 ms
64 bytes from 10.33.0.121: icmp_seq=65 ttl=61 time=62.4 ms
[...]
\end{lstlisting}


\section{Use-Case 2: Road Warrior} \label{Use-Case 2: Road Warrior}
%Anforderung: Ende-zu-Ende-Erreichbarkeit! Für z.B. Endpoint Scanning
%Gleiches Profil für alle VPN-Konzentratoren



\subsection{Probleme und Lösungsfindung}
%Import von Remote State, um Azure VPN Gateway zu verkürzen... Eigentlich ziemlich coole Idee von mir... :)
%Statische Routen innerhalb VPC nicht more specific -> Transit Gateway
%\section{Use-Case 3: Server-zu-Server} \label{Use-Case 3: Server-zu-Server}
%%\include{Chapters/05_Evaluation}
%%\include{Chapters/06_Fazit_und_Ausblick}
%\input{Chapters/09_Examples/Example}
\pagestyle{plain} 

%Literaturverzeichnis
\newpage
\bibliographystyle{unsrt}
\raggedright
\bibliography{biblatex}

\listoffigures
\addcontentsline{toc}{chapter}{Listingverzeichnis}
\listoflistings

\end{document}
