\chapter{Abstract}

Unternehmen verlagern zunehmend digitale Geschäftsprozesse, die zuvor auf eigener, physischer Infrastruktur abgebildet wurden, hin zu \textit{Public Cloud}-Providern wie Amazon Web Services und Microsoft Azure. Von dieser Verlagerung in die Cloud erhoffen sich die Unternehmen Kosteneinsparungen, eine vereinfachte Skalierbarkeit und eine hohe Verfügbarkeit.\\
Ein kompletter Umzug zu Public Cloud-Anbietern findet aber selten statt. Viele Unternehmen betreiben weiter eine eigene IT-Infrastruktur mit exklusiven, eigenen Rechenressourcen, z.B. für die Verarbeitung von sensiblen Daten. Dabei spricht man von einer \textit{Private Cloud}. Werden in einem Unternehmen Private und Public Cloud-Lösungen miteinander verbunden, spricht man von einer \textit{Hybrid Cloud}. \\
Diese Arbeit zeigt, wie eine Hybrid Cloud aufgebaut werden kann, um einen vom Internet isolierten Datenaustausch via IPv4 zu gewährleisten, mit einem Fokus auf kleine bis mittelgroße Skalierungen. Dies passiert mit Hilfe von offenen Internetstandards ohne Zuhilfenahme von proprietären Herstellerlösungen.\\
Die Fragestellung wird anhand zweier Use Cases analysiert, welche praxisrelevante Hybrid Cloud-Szenarien in Form eines Proof of Concepts darstellen. Use Case 1 bildet die grundlegende IPv4-Kommunikation zwischen verschiedenen Cloud-Plattformen ab (\glqq Ende-zu-Ende-Konnektivität\grqq{}). Darauf aufbauend erfolgt in Use Case 2 eine latenzarme Anbindung von mobilen Endgeräten zum nächstgelegenen (Cloud-)Standort (\glqq Roadwarrior-VPN\grqq{}). Der Aufbau und die Konfiguration der Hybrid Cloud werden durch Werkzeuge wie IP Address Management und Terraform automatisiert. Anhand von Evaluationskriterien werden die Lösungen bewertet und Verbesserungspotentiale aufgezeigt.\\
Weiterführende Themen im Hybrid Cloud-Umfeld wie Bandbreitenoptimierung werden identifiziert. Analoge Szenarien für große Hybrid Cloud-Skalierungen werden im Ausblick diskutiert.