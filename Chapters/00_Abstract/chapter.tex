\chapter{Abstract}
Unternehmen verlagern zunehmend digitale Prozesse, die ehemals auf einer eigenen, physischen Infrastruktur abgebildet wurden, hin zu \textit{Public Cloud}-Providern wie Amazon Web Services und Microsoft Azure. Die Cloud im Allgemeine verspricht ein schnelles Deployment und hohe Verfügbarkeiten von Internetdiensten. Weiterhin können diese Dienste dynamisch zum Anfrageaufkommen skaliert werden. Public Clouds bieten Unternehmen den Vorteil, dass keine physische Infrastruktur vonnöten ist, um diese Dienste anzubieten. Ein kompletter Umzug findet dabei aber oftmals nicht statt und es verbleibt Infrastruktur in einer \textit{Private Cloud}, z.B. um die Speicherung von sensitiven Daten weiterhin kontrollieren zu können.\\
Das Ziel der vorliegenden Arbeit ist zu zeigen, wie eine Vernetzung von Public und Private Cloud ermöglicht werden kann, um einen vom Internet isolierten Datenaustausch via IPv4 zu gewährleisten. Hauptsächlicher Anwendungsbereich sind kleine bis mittelgroße Skalierungen.\\
Um diese Forschungsfrage zu beantworten, wurden zwei Use-Cases definiert, welche praxisrelevante Hybrid Cloud-Szenarien widerspiegeln. Use-Case 1 stellt die grundlegende IPv4-Kommunikation (\glqq Ende-zu-Ende\grqq{}) zwischen verschiedenen Cloud-Plattformen dar, während in Use-Case 2 (\glqq Roadwarrior-VPN\grqq{}) eine latenzarme Verbindung von Endgeräten zu der nächstgelegenen Cloud-Plattform erfolgen soll. Durch geeignete Werkzeuge soll dies automatisiert geschehen.\\
Mit Hilfe geeigneter Evaluationskriterien konnte erfolgreich gezeigt werden, dass eine technische Machbarkeit der aufgezeigten Szenarien grundsätzlich gegeben ist. Technische Hürden existieren und werden ausführlich beleuchtet.\\
Weiterführende Forschungsthemen im Hybrid Cloud-Umfeld wie Bandbreitenoptimierung sind erdenklich und werden im Ausblick der Arbeit näher erläutert.\\