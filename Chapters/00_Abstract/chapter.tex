\chapter{Abstract}

Unternehmen verlagern zunehmend digitale Geschäftsprozesse, die zuvor auf eigener, physischer Infrastruktur abgebildet wurden, hin zu \textit{Public Cloud}-Providern wie Amazon Web Services und Microsoft Azure. Von der Verlagerung in die Cloud erhoffen sich die Unternehmen Kosteneinsparungen, eine vereinfachte Skalierbarkeit und eine hohe Verfügbarkeit.\\
Ein kompletter Umzug zu Public Cloud-Anbietern findet aber selten statt. Viele Unternehmen betreiben weiter eine eigene IT-Infrastruktur mit exklusiven Rechenressourcen, z.B. für die Verarbeitung von sensiblen Daten. Dabei spricht man von einer \textit{Private Cloud}. Werden in einem Unternehmen Private und Public Cloud-Lösungen miteinander vernetzt, spricht man von einer \textit{Hybrid Cloud}. \\
Diese Arbeit zeigt, wie eine Hybrid Cloud aufgebaut werden kann, um einen vom Internet isolierten Datenaustausch via IPv4 zu gewährleisten, mit einem Fokus auf kleine bis mittelgroße Skalierungen.\\
Die Fragestellung wurde anhand zweier Use Cases analysiert, welche praxisrelevante Hybrid Cloud-Szenarien widerspiegeln. Use Case 1 stellt die grundlegende IPv4-Kommunikation (\glqq Ende-zu-Ende\grqq{}) zwischen verschiedenen Cloud-Plattformen dar. Darauf aufbauend erfolgt in Use Case 2 (\glqq Roadwarrior-VPN\grqq{}) eine latenzarme Verbindung von Endgeräten zu der nächstgelegenen Cloud-Plattform. Der Aufbau und die Konfiguration der Hybrid Cloud wurde durch Werkzeuge automatisiert.\\
Beide Use Cases wurden erfolgreich als Proof of Concepts (PoC) umgesetzt. Anhand von Evaluationskriterien werden die Lösungen bewertet und Verbesserungspotentiale aufgezeigt.\\
Weiterführende Themen im Hybrid Cloud-Umfeld wie Bandbreitenoptimierung werden identifiziert und im Ausblick der Arbeit näher erläutert. Zudem wird diskutiert, welche weiteren Schritte für eine produktive Lösung nötig sind.\\