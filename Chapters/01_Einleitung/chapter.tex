%Alle Codebeispiele finden sich im Github
%SASE? https://en.wikipedia.org/wiki/Secure_Access_Service_Edge
blub

\ifFalse
Serifen Schrit für Programmnamen?
Teilweise gekürzt mit [...] oder Funktionsnamen der Einfachheit umbenannt oder vereinfacht dargestellt
RFC-Doku-Addressen
Getestet mit Terraform Version...
Abkürzungen AWS, VPC werden einmal erläutert, danach Schicht im Schacht
Abgrenzung: Ende-zu-Ende-Konnektivität gegeben (wegen Amazon NAT)
Rein technischer Natur, es wird nicht auf Kostenoptimierung eingangen -> dafür ist Cloud-Costs zu komplex
Abgrenzung: alle Public-IP und Private-IP beziehen sich auf IPv3!!!
Kommandos sind mit Dollar markiert. Sie werden genutzt, wenn ein Filterausdruck erkennbar sein soll, oftmals mit grep
Typische Schalter in dieser Arbeit:
-A n: Zeige auch die nächsten n Zeilen nach einem Treffer
-B n: Zeige auch die vorherigen n Zeilen vor einem Treffer
-o Zeige ausschließlich den Treffer, nicht die komplette Zeile, oftmals im Zusammenspiel mit -E
-E Suche mit Hilfe von regulären Ausdrücken (Regex)
-v ignoriere Zeilen, die das Pattern beinhalten (invertiere)
Linux-Grundlagen werden vorausgesetzt, iptables, cat, head, tail, 
Alle Zeichnungen wurden draw.io gemacht. Auch die Shapes entstammen draw.io
Monospace für Programmaufrufe
Unterstrichen für Dateinamen
\fi