\newglossaryentry{Deployment}
{
        name=Deployment,
        description={Bereitstellung von Infrastruktur in Form von Hard- und Software}
}
\newglossaryentry{OSI-Schichtenmodell}
{
        name=OSI-Schichtenmodell,
        description={Das Open Systems Interconnection-Schichtenmodell stellt verschiedene Funktionen und Schnittstellen in der Netzwerkvermittlung abstrahiert dar in Schichten 1 bis 7}
}
\newglossaryentry{Client}
{
        name=Client,
        description={Endgerät in einem Netzwerk, welches Netzwerkverbindungen initiiert}
}
\newglossaryentry{Client-to-Site}
{
        name=Client-to-Site,
        description={VPN-Verbindung von einem Client zu einem Standort}
}
\newglossaryentry{Site-to-Site}
{
        name=Site-to-Site,
        description={VPN-Verbindung zwischen zwei Standorten}
}
\newglossaryentry{Rollback}
{
        name=Rollback,
        description={Zurücksetzen auf einen alten (funktionierenden) Zustand eines IT-Systems}
}
\newglossaryentry{on-premises}
{
        name=on-premises,
        description={Betrieb von IT-Infrastruktur im eigenen Rechenzentrum}
}
\newglossaryentry{Roadwarrior}
{
        name=Roadwarrior,
        description={Mobiler Mitarbeiter mit unbekannter IP mit Zugriff auf interne Ressourcen via VPN}
}
\newglossaryentry{GeoIP}
{
        name=GeoIP,
        description={Datenbank für Ermittlung von Standorten anhand von IP-Adressen}
}
\newglossaryentry{View}{
        name=View,
        description={Verschiedene Ansichten auf DNS-Zonen, z.B. anhand von Absender-IP-Adresse}
}
\newglossaryentry{VPN-Konzentrator}
{
        name=VPN-Konzentrator,
        description=Gegenstelle für VPN-Clients (Client-to-Site)
}
\newglossaryentry{VPN-Gateway}
{
        name=VPN-Gateway,
        description=Gegenstelle für VPN-Standorte (Site-to-Site)
}
\newglossaryentry{SSH}
{
        name=Secure Shell (SSH),
        description=Verschlüsseltes Protokoll für Fernzugriffe auf Befehlszeilen
} 
\newglossaryentry{iptables}
{
        name=iptables,
        description=Paketfilter für Linux
}
\newglossaryentry{Fallback}
{
        name=Fallback,
        description=Zurückfallen auf einen alternativen (Routing-)Pfad
} 
\newacronym{CIDR}{CIDR}{Classless Inter-Domain Routing}
\newacronym{DNS}{DNS}{Domain Name System}
\newacronym{FQDN}{FQDN}{Fully Qualified Domain Name}
\newacronym{ICMP}{ICMP}{Internet Control Message Protocol}
\newacronym{TSIG}{TSIG}{Transaction SIGnature}
\newacronym{ACL}{ACL}{Access Control List}
\newacronym{SG}{SG}{(AWS) Security Group}
\newacronym{NSG}{NSG}{(Azure) Network Security Group}
\newacronym{VPN}{VPN}{Virtual Private Network}
\newacronym{IPsec}{IPsec}{IP Protocol Security}
\newacronym{PKI}{PKI}{Public Key Infrastructure}
\newacronym{BGP}{BGP}{Border Gateway Protocol}
\newacronym{VPC}{VPC}{(AWS) Virtual Private Cloud}
\newacronym{VNET}{VNET}{(Azure) Virtual Network}
\newacronym{NAT}{NAT}{Network Address Translation}
\newacronym{TCP}{TCP}{Transmission Control Protocol}
\newacronym{UDP}{UDP}{User Datagram Protocol}
\newacronym{MSS}{MSS}{Maximum Segment Size}
\newacronym{RFC}{RFC}{Request for Comments}
\newacronym{VM}{VM}{Virtuelle Maschine}
\newacronym{AS}{AS}{Autonomes System}
\newacronym{CM}{CM}{Configuration Management}
\newacronym{IPAM}{IPAM}{IP Address Management}
\newacronym{SD-WAN}{SD-WAN}{Software Defined WAN}
\newacronym{MTU}{MTU}{Maximum Transmission Unit}
\glsunsetall
